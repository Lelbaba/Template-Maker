\newcommand\TEAM{IUT SuperSonic}
\newcommand\UNI{Islamic University of Technology}
\newcommand\COLS{3}
\newcommand\ORT{portrait}
\newcommand\FSZ{10}
\documentclass[FSZ,a4paper,onesided]{article}
\usepackage[utf8]{inputenc}
\usepackage{amsmath}
\usepackage{listings}
\usepackage{graphicx}
\usepackage{multicol}
\usepackage[utf8]{inputenc}
\usepackage[english]{babel}
\usepackage[usenames,dvipsnames]{color}
\usepackage{verbatim}
\usepackage{hyperref}
\usepackage{geometry}
\usepackage{fancyhdr}
\usepackage{titlesec}

\geometry{verbose,\ORT,a4paper,tmargin=1.0cm,bmargin=.5cm,lmargin=.5cm,rmargin=.5cm, headsep=.5cm}

\definecolor{dkgreen}{rgb}{0,0.6,0}
\definecolor{gray}{rgb}{0.5,0.5,0.5}
\definecolor{mauve}{rgb}{0.58,0,0.82}

\lstset{frame=tb,
  language=C++,
  aboveskip=1mm,
  belowskip=1mm,
  showstringspaces=false,
  columns=flexible,
  basicstyle={\footnotesize\ttfamily},
  numbers=none,
  numberstyle=\tiny\color{gray},
  keywordstyle=\color{blue},
  commentstyle=\color{dkgreen},
  stringstyle=\color{mauve},
  breaklines=true,
  breakatwhitespace=false,
  tabsize=1
}

\fancyhf{}
\renewcommand{\headrulewidth}{1pt}
\pagestyle{fancy}
\lhead{\large{\textbf{\TEAM}, \textbf{\UNI}}}
\rhead{\thepage}

\titleformat*{\subsection}{\large\bfseries}
\titleformat*{\subsubsection}{\large\bfseries}
\titlespacing*{\section}
{0pt}{0ex}{0ex}
\titlespacing*{\subsection}
{0pt}{0ex}{0ex}
\titlespacing*{\subsubsection}
{0pt}{0ex}{0ex}
\setlength{\columnsep}{0.05in}
\setlength{\columnseprule}{1px}


\begin{document}

\begin{multicols*}{\COLS}
\pagenumbering{gobble}
\tableofcontents
\newpage
\pagenumbering{arabic}
\lstloadlanguages{C++,Java}
\subsection*{Sublime Build}
\begin{lstlisting}
{
    "cmd" : ["g++ -std=c++14 -DSONIC $file_name -o $file_base_name && timeout 4s ./$file_base_name<inputf.in>outputf.in"], 
    "selector" : "source.cpp",
    "file_regex": "^(..[^:]*):([0-9]+):?([0-9]+)?:? (.*)$",
    "shell": true,
    "working_dir" : "$file_path"
}
\end{lstlisting}


\section{All Macros}
\begin{lstlisting}
//#pragma GCC optimize("Ofast")
//#pragma GCC optimization ("O3")
//#pragma comment(linker, "/stack:200000000")
//#pragma GCC optimize("unroll-loops")
//#pragma GCC target("sse,sse2,sse3,ssse3,sse4,popcnt,abm,mmx,avx,tune=native")
#include <ext/pb_ds/assoc_container.hpp>
#include <ext/pb_ds/tree_policy.hpp>
using namespace __gnu_pbds;
    //find_by_order(k) --> returns iterator to the kth largest element counting from 0
    //order_of_key(val) --> returns the number of items in a set that are strictly smaller than our item
template <typename DT> 
using ordered_set = tree <DT, null_type, less<DT>, rb_tree_tag,tree_order_statistics_node_update>;

int kx[] =              {-2,-2,-1,+1,+2,+2,+1,-1};
int ky[] =              {-1,+1,+2,+2,+1,-1,-2,-2};

#define fastio            ios_base::sync_with_stdio(0);cin.tie(0);

#define Make(x,p)       (x | (1<<p))
#define DeMake(x,p)     (x & ~(1<<p))
#define Check(x,p)      (x & (1<<p))

#define DEBUG(x)        cerr << #x << " = " << x << endl

template<size_t N>
bitset<N> rotl( std::bitset<N> const& bits, unsigned count ) {
    count %= N;  // Limit count to range [0,N)
    return bits << count | bits >> (N - count);
}\end{lstlisting}
\section{DP}
\subsection{Convex Hull Trick}
\begin{lstlisting}
 struct line {
   ll m, c;
   line() {}
   line(ll m, ll c) : m(m), c(c) {}
 };
 struct convex_hull_trick {
   vector<line>lines;
   int ptr = 0;
   convex_hull_trick() {}
   bool bad(line a, line b, line c) {
     return 1.0 * (c.c - a.c) * (a.m - b.m) < 1.0 * (b.c - a.c) * (a.m - c.m);
   }
   void add(line L) {
     int sz = lines.size();
     while (sz >= 2 && bad(lines[sz - 2], lines[sz - 1], L)) {
       lines.pop_back(); sz--;
     }
     lines.pb(L);
   }
   ll get(int idx, int x) {
     return (1ll * lines[idx].m * x + lines[idx].c);
   }
   ll query(int x) {
     if (lines.empty()) return 0;
     if (ptr >= lines.size()) ptr = lines.size() - 1;
     while (ptr < lines.size() - 1 && get(ptr, x) > get(ptr + 1, x)) ptr++;
     return get(ptr, x);
   }
 };
 ll sum[MAX];
 ll dp[MAX];
 int arr[MAX];
 int main() {
   fastio;
   int t;
   cin >> t;
   while (t--) {
     int n, a, b, c;
     cin >> n >> a >> b >> c;
     for (int i = 1; i <= n; i++) cin >> sum[i];
     for (int i = 1; i <= n; i++) dp[i] = 0, sum[i] += sum[i - 1];
     convex_hull_trick cht;
     cht.add( line(0, 0) );
     for (int pos = 1; pos <= n; pos++) {
       dp[pos] = cht.query(sum[pos]) - 1ll * a * sqr(sum[pos]) - c;
       cht.add( line(2ll * a * sum[pos], dp[pos] - a * sqr(sum[pos])) );
     }
     ll ans = (-1ll * dp[n]);
     ans += (1ll * sum[n] * b);
     cout << ans << "\n";
   }
 } 
\end{lstlisting}
\subsection{Divide and Conquer DP}
\begin{lstlisting}
inline void compute(int cur, int L, int R, int best_L, int best_R) {
  if (L > R) return;
  int mid = (L + R) >> 1;
  pair<ll, int>best = {inf, -1};
  for (int k = best_L; k <= min(best_R, mid); k++) {
    best = min(best, {dp[cur ^ 1][k - 1] + getCost(k, mid), k});
  }
  dp[cur][mid] = best.ff;
  int best_id = best.ss;
  compute(cur, L, mid - 1, best_L, best_id);
  compute(cur, mid + 1, R, best_id, best_R);
}
// in main
int cur = 0;
for (int i = 1; i <= n; i++) dp[1][i] = inf;
for (int guard = 1; guard <= g; guard++) {
  compute(cur, 1, n, 1, n); cur ^= 1;
}
ll ans = dp[cur ^ 1][n];
\end{lstlisting}
\subsection{Knuth Iterative}
\begin{lstlisting}
for (int i = 1; i <= n; i++) {
  path[i][i] = i;
  dp[i][i] = 0;
}

for (int len = 2; len <= n; len++) {
  for (int st = 1; st + len - 1 <= n; st++) {
    int ed = st + len - 1;
    int L = max(st, path[st][ed - 1]);
    int R = min(ed - 1, path[st + 1][ed]);
    dp[st][ed] = INT_MAX;
    for (int i = L; i <= R; i++) {
      int cur = dp[st][i] + dp[i + 1][ed] + arr[ed] - arr[st - 1];
      if (dp[st][ed] > cur) {
        dp[st][ed] = cur;
        path[st][ed] = i;
      }
    }
  }
}
cout << dp[1][n] << "\n";\end{lstlisting}
\subsection{Knuth Optimization}
\begin{lstlisting}
ll solve(int st, int ed) { ///recursive
  if (st == ed) {
    path[st][ed] = st;
    return 0;
  }
  ll &ret = dp[st][ed];
  if (ret != -1) return ret;
  solve(st, ed - 1); solve(st + 1, ed);
  int L = max(st, path[st][ed - 1]);
  int R = min(ed - 1, path[st + 1][ed]);
  ret = inf;
  for (int i = L; i <= R; i++) {
    ll cur = solve(st, i) + solve(i + 1, ed);
    cur += (arr[ed] - arr[st - 1]);
    if (cur < ret) ret = cur; path[st][ed] = i;
  }
  return ret;
}
///knuth for divide and conquer
int solve(int group, int pos) {
  if (!pos) return dp[group][pos] = 0;
  if (!group) return dp[group][pos] = INT_MAX;
  int &ret = dp[group][pos];
  if (ret != -1) return ret;
  int L = 1, R = pos;
  if (pos - 1 > 0) {
    solve(group, pos - 1);
    L = max(L, path[group][pos - 1]);
  }
  if (group + 1 <= m) {
    solve(group + 1, pos);
    R = min(R, path[group + 1][pos]);
  }
  ret = INT_MAX;
  for (int i = L; i <= R; i++) {
    int cur = solve(group - 1, i - 1) + 1ll * (arr[pos] - arr[i]) * (arr[pos] - arr[i]);
    if (cur < ret) {
      ret = cur;
      path[group][pos] = i;
    }
  }
  return ret;
}
\end{lstlisting}
\subsection{Li Chao Tree}
\begin{lstlisting}
struct line {
  ll m, c;
  line(ll m = 0, ll c = 0) : m(m), c(c) {}
};
ll calc(line L, ll x) {
  return 1ll * L.m * x + L.c;
}
struct node {
  ll m, c;
  line L;
  node *lft, *rt;
  node(ll m = 0, ll c = 0, node *lft = NULL, node *rt = NULL) : L(line(m, c)), lft(lft), rt(rt) {}
};
struct LiChao {
  node *root;
  LiChao() {
    root = new node();
  }
  void update(node *now, int L, int R, line newline) {
    int mid = L + (R - L) / 2;
    line lo = now->L, hi = newline;
    if (calc(lo, L) > calc(hi, L)) swap(lo, hi);
    if (calc(lo, R) <= calc(hi, R)) {
      now->L = hi;
      return;
    }
    if (calc(lo, mid) < calc(hi, mid)) {
      now->L = hi;
      if (now->rt == NULL) now->rt = new node();
      update(now->rt, mid + 1, R, lo);
    } else {
      now->L = lo;
      if (now->lft == NULL) now->lft = new node();
      update(now->lft, L, mid, hi);
    }
  }
  ll query(node *now, int L, int R, ll x) {
    if (now == NULL) return -inf;
    int mid = L + (R - L) / 2;
    if (x <= mid) return max( calc(now->L, x), query(now->lft, L, mid, x) );
    else return max( calc(now->L, x), query(now->rt, mid + 1, R, x) );
  }
};
\end{lstlisting}
\subsection{Number Permutation}
\begin{lstlisting}
ll dp[2][3005]; ll sum[2][3005];
int dir[3005];
int arr[MAX];
int main() {
  fastio;
  int n;
  string s;
  cin >> n >> s;
  s = '#' + s;
  s.pb('<'); ///last element less than the element placed after it
  sum[1][0] = 1;
  int cur = 0;
  for (int baki = 1; baki <= n; baki++) {
    if (s[baki] == '<') dp[cur][0] = 0;
    else dp[cur][0] = sum[cur ^ 1][baki - 1];
    for (int small = 1; small <= baki; small++) {
      if (s[baki] == '<') dp[cur][small] = sum[cur ^ 1][small - 1];
      else {
        int big = baki - small;
        dp[cur][small] = sum[cur ^ 1][small + big - 1];
        dp[cur][small] -= sum[cur ^ 1][small - 1];
        if (dp[cur][small] < 0) dp[cur][small] += MOD;
      }
    }
    sum[cur][0] = dp[cur][0];
    for (int small = 1; small <= baki; small++) {
      sum[cur][small] = (sum[cur][small - 1] + dp[cur][small]);
      if (sum[cur][small] >= MOD) sum[cur][small] -= MOD;
    }
    cur ^= 1;
  }
  ll ans = dp[cur ^ 1][n];
  cout << ans << "\n";
}

\end{lstlisting}
\subsection{Same Color Group}
\begin{lstlisting}
int prv[21]; ll cost[21][21];
ll dp[1 << 21]; int m, n;
bool ok[1 << 21];
ll solve(ll mask) {
  if (mask == (1 << m) - 1) return 0ll;
  ll &ret = dp[mask];
  if (ok[mask]) return ret;
  ok[mask] = true; ret = inf;
  for (int i = 0; i < m; i++) {
    if (!(mask & (1 << i) )) {
      ll c = 0;
      for (int j = 0; j < m; j++) {
        if ((mask & (1 << j)))
          c += cost[i][j];
      }
      ret = min(ret, c + solve((mask | (1 << i))));
    }
  }
  return ret;
}
int arr[MAX];
int main() {
  for (int i = 0; i < n; i++) {
    int val = arr[i];
    val--; prv[val]++;
    for (int j = 0; j < m; j++) {
      if (val == j) continue;
      cost[val][j] += prv[j];
    }
  }
  ll ans = solve(0);
}

\end{lstlisting}
\subsection{Triangulation DP}
\begin{lstlisting}
bool valid[205][205];
ll dp[205][205];
ll solve(int L, int R) {
  if (L + 1 == R) return 1;
  if (dp[L][R] != -1) return dp[L][R];
  ll ret = 0;
  for (int mid = L + 1; mid < R; mid++) {
    if (valid[L][mid] && valid[mid][R]) {
      ///selecting triangle(P[L], P[mid], P[R])
      ll temp = ( solve(L, mid) * solve(mid, R) ) % MOD;
      ret = (ret + temp) % MOD;
    }
  }
  return dp[L][R] = ret;
}
\end{lstlisting}
\section{Data Structures}
\subsection{DSU on Tree}
\begin{lstlisting}
///Query: Number of distinct names among all the k'th son of a node.
const int N = 100005;
string name[N];
vector<int>G[N];
vector<pii>Q[N];
int L[N],ans[N];

void dfs(int v,int d){
    L[v]=d;
    for(int i:G[v]) dfs(i,d+1);
    return;
}

void dsu(int v,map<int,set<string>>&mp){
    for(int i:G[v]){
        map<int,set<string>>s;
        dsu(i,s);
        if(s.size()>mp.size()) swap(mp,s);
        for(auto it:s) mp[it.ff].insert(all(it.ss));
    }
    if(v!=0) mp[L[v]].insert(name[v]); //Here zero is not a actual node
    for(pii p:Q[v]) ans[p.ss] = mp[p.ff].size();
    return;
}

int main(){
    int n;
    cin >> n;
    FOR(i,1,n){
        int u;
        cin >> name[i] >> u;
        G[u].pb(i);
    }
    dfs(0,0);
    int q;
    cin >>q;
    FOR(i,1,q){
        int v,k;
        cin >> v >> k;
        Q[v].pb(pii(k+L[v],i)); //Actual level
    }
    map<int,set<string>>mp;
    dsu(0,mp);
    FOR(i,1,q) cout << ans[i] << '\n';
    return 0;
}
\end{lstlisting}
\subsection{Dominator Tree}
\begin{lstlisting}
struct dominator {
  int n, d_t;
  vector<vector<int>> g, rg, tree, bucket;
  vector<int> sdom, dom, par, dsu, label, val, rev;
  dominator() {}
  dominator(int n) : 
    n(n), d_t(0), g(n + 1), rg(n + 1),
    tree(n + 1), bucket(n + 1), sdom(n + 1),
    dom(n + 1), par(n + 1), dsu(n + 1),
    label(n + 1), val(n + 1), rev(n + 1)
  { for (int i = 1; i <= n; i++) sdom[i] = dom[i] = dsu[i] = label[i] = i; }

  void add_edge(int u, int v) { g[u].pb(v); }
  int dfs(int u) {
    d_t++;
    val[u] = d_t, rev[d_t] = u;
    label[d_t] = sdom[d_t] = dom[d_t] = d_t;
    for (int v : g[u]) {
      if (!val[v]) {
        dfs(v);
        par[val[v]] = val[u];
      }
      rg[val[v]].pb(val[u]);
    }
  }
  int findpar(int u, int x = 0) {
    if (dsu[u] == u) return x ? -1 : u;
    int v = findpar(dsu[u], x + 1);
    if (v < 0) return u;
    if (sdom[label[dsu[u]]] < sdom[label[u]]) label[u] = label[dsu[u]];
    dsu[u] = v;
    return x ? v : label[u];
  }
  void join(int u, int v) { dsu[v] = u; }
  vector<vector<int>> build(int s) {
    dfs(s);
    for (int i = n; i >= 1; i--) {
      for (int j = 0; j < rg[i].size(); j++) {
        sdom[i] = min(sdom[i], sdom[ findpar(rg[i][j]) ]);
      }
      if (i > 1) bucket[sdom[i]].pb(i);
      for (int w : bucket[i]) {
        int v = findpar(w);
        if (sdom[v] == sdom[w]) dom[w] = sdom[w];
        else dom[w] = v;
      }
      if (i > 1) join(par[i], i);
    }
    for (int i = 2; i <= n; i++) {
      if (dom[i] != sdom[i]) dom[i] = dom[dom[i]];
      tree[rev[i]].pb(rev[dom[i]]);
      tree[rev[dom[i]]].pb(rev[i]);
    }
    return tree;
  }
};
\end{lstlisting}
\subsection{Hopcroft Karp}
\begin{lstlisting}
#include<bits/stdc++.h>
using namespace std;

const int N = 3e5 + 9;

struct HopcroftKarp {
  static const int inf = 1e9;
  int n;
  vector<int> l, r, d;
  vector<vector<int>> g;
  HopcroftKarp(int _n, int _m) {
    n = _n;
    int p = _n + _m + 1;
    g.resize(p);
    l.resize(p, 0);
    r.resize(p, 0);
    d.resize(p, 0);
  }
  void add_edge(int u, int v) {
    g[u].push_back(v + n); //right id is increased by n, so is l[u]
  }
  bool bfs() {
    queue<int> q;
    for (int u = 1; u <= n; u++) {
      if (!l[u]) d[u] = 0, q.push(u);
      else d[u] = inf;
    }
    d[0] = inf;
    while (!q.empty()) {
      int u = q.front();
      q.pop();
      for (auto v : g[u]) {
        if (d[r[v]] == inf) {
          d[r[v]] = d[u] + 1;
          q.push(r[v]);
        }
      }
    }
    return d[0] != inf;
  }
  bool dfs(int u) {
    if (!u) return true;
    for (auto v : g[u]) {
      if(d[r[v]] == d[u] + 1 && dfs(r[v])) {
        l[u] = v;
        r[v] = u;
        return true;
      }
    }
    d[u] = inf;
    return false;
  }
  int maximum_matching() {
    int ans = 0;
    while (bfs()) {
      for(int u = 1; u <= n; u++) if (!l[u] && dfs(u)) ans++;
    }
    return ans;
  }
};
int32_t main() {
  ios_base::sync_with_stdio(0);
  cin.tie(0);
  int n, m, q;
  cin >> n >> m >> q;
  HopcroftKarp M(n, m);
  while (q--) {
    int u, v;
    cin >> u >> v;
    M.add_edge(u, v);
  }
  cout << M.maximum_matching() << '\n';
  return 0;
}
\end{lstlisting}
\subsection{Implicit Segment Tree}
\begin{lstlisting}
struct node {
  int val;
  node *lft, *rt;
  node() {}
  node(int val = 0) : val(val), lft(NULL), rt(NULL) {}
};

struct implicit_segtree {
  node *root;
  implicit_segtree() {}
  implicit_segtree(int n) {
    root = new node(n);
  }
  void update(node *now, int L, int R, int idx, int val) {
    if (L == R) {
      now -> val += val;
      return;
    }
    int mid = L + (R - L) / 2;
    if (now->lft == NULL) now->lft = new node(mid - L + 1);
    if (now->rt == NULL) now->rt = new node(R - mid);
    if (idx <= mid) update(now->lft, L, mid, idx, val);
    else update(now->rt, mid + 1, R, idx, val);
    now->val = (now->lft)->val + (now->rt)->val;
  }

  int query(node *now, int L, int R, int k) {
    if (L == R) return L;
    int mid = L + (R - L) / 2;
    if (now->lft == NULL) now->lft = new node(mid - L + 1);
    if (now->rt == NULL) now->rt = new node(R - mid);
    if (k <= (now->lft)->val) return query(now->lft, L, mid, k);
    else return query(now->rt, mid + 1, R, k - (now->lft)->val);
  }
};
\end{lstlisting}
\subsection{Implicit Treap}
\begin{lstlisting}
mt19937 rnd(chrono::steady_clock::now().time_since_epoch().count());
typedef struct node* pnode;
struct node {
  int prior, sz;
  ll val, sum, lazy;
  bool rev;
  node *lft, *rt;
  node(int val = 0, node *lft = NULL, node *rt = NULL) : lft(lft), rt(rt), prior(rnd()), sz(1), val(val), rev(false), sum(0), lazy(0) {}
};
struct implicit_treap {
  pnode root;
  implicit_treap() {
    root = NULL;
  }
  int get_sz(pnode now) {
    return now ? now->sz : 0;
  }
  void update_sz(pnode now) {
    if (!now) return;
    now->sz = 1 + get_sz(now->lft) + get_sz(now->rt);
  }
  // lazy sum
  void push(pnode now) {
    if (!now || !now->lazy) return;
    now->val += now->lazy;
    now->sum += get_sz(now) * now->lazy;
    if (now->lft) now->lft->lazy += now->lazy;
    if (now->rt) now->rt->lazy += now->lazy;
    now->lazy = 0;
  }
  void combine(pnode now) {
    if (!now) return;
    now->sum = now->val; // reset the node
    push(now->lft), push(now->rt); // update lft and rt
    now->sum += (now->lft ? now->lft->sum : 0) + (now->rt ? now->rt->sum : 0);
  }
  // reverse substring
  void push(pnode now) {
    if (!now || !now->rev) return;
    now->rev = false;
    swap(now->lft, now->rt);
    if (now->lft) now->lft->rev ^= true;
    if (now->rt) now->rt->rev ^= true;
  }
  sort ascending or descending
  void push(pnode now) {
    if (!now || !now->sort_kor) return;
    if (now->sort_kor == -1) swap(now->lft, now->rt);
    int cnt[26];
    for (int i = 0; i < 26; i++) cnt[i] = now->cnt[i];
    int idx = 0;
    if (now->lft) {
      memset(now->lft->cnt, 0, sizeof now->lft->cnt);
      int lft_sz = get_sz(now->lft);
      while (idx < 26 && lft_sz) {
        int mn = min(cnt[idx], lft_sz);
        now->lft->cnt[idx] = mn;
        cnt[idx] -= mn; lft_sz -= mn;
        if (!cnt[idx]) idx++;
      }
      now->lft->sort_kor = now->sort_kor;
    }
    while (!cnt[idx]) idx++;
    now->val = idx, cnt[idx]--;
    if (!cnt[idx]) idx++;
    if (now->rt) {
      memset(now->rt->cnt, 0, sizeof now->rt->cnt);
      int rt_sz = get_sz(now->rt);
      while (idx < 26 && rt_sz) {
        int mn = min(cnt[idx], rt_sz);
        now->rt->cnt[idx] = mn;
        cnt[idx] -= mn; rt_sz -= mn;
        if (!cnt[idx]) idx++;
      }
      now->rt->sort_kor = now->sort_kor;
    }
    if (now->sort_kor == -1) swap(now->lft, now->rt);
    now->sort_kor = 0;
  }
  void combine(pnode now) {
    if (!now) return;
    memset(now->cnt, 0, sizeof now->cnt);
    for (int i = 0; i < 26; i++) {
      now->cnt[i] = (now->lft ? now->lft->cnt[i] : 0) + (now->rt ? now->rt->cnt[i] : 0);
    }
    now->cnt[now->val]++;
  }
  ///first pos ta elements go to left, others go to right
  void split(pnode now, pnode &lft, pnode &rt, int pos, int add = 0) {
    if (!now) return void(lft = rt = NULL);
    push(now);
    int cur = add + get_sz(now->lft);
    if (cur < pos) split(now->rt, now->rt, rt, pos, cur + 1), lft = now;
    else split(now->lft, lft, now->lft, pos, add), rt = now;
    update_sz(now); combine(now);
  }
  void merge(pnode &now, pnode lft, pnode rt) {
    push(lft);
    push(rt);
    if (!lft || !rt) now = lft ? lft : rt;
    else if (lft->prior > rt->prior) merge(lft->rt, lft->rt, rt), now = lft;
    else merge(rt->lft, lft, rt->lft), now = rt;
    update_sz(now); combine(now);
  }
  void insert(int pos, ll val) {
    if (!root) return void(root = new node(val));
    pnode lft, rt;
    split(root, lft, rt, pos - 1);
    pnode notun = new node(val);
    merge(root, lft, notun);
    merge(root, root, rt);
  }
  void erase(int pos) {
    pnode lft, rt, temp;
    split(root, lft, rt, pos);
    split(lft, lft, temp, pos - 1);
    merge(root, lft, rt);
    delete(temp);
  }
  void reverse(int l, int r) {
    pnode lft, rt, mid;
    split(root, lft, mid, l - 1);
    split(mid, mid, rt, r - l + 1);
    mid->rev ^= true;
    merge(root, lft, mid);
    merge(root, root, rt);
  }
  void right_shift(int l, int r) {
    pnode lft, rt, mid, last;
    split(root, lft, mid, l - 1);
    split(mid, mid, rt, r - l + 1);
    split(mid, mid, last, r - l);
    merge(mid, last, mid);
    merge(root, lft, mid);
    merge(root, root, rt);
  }
  void output(pnode now, vector<int>&v) {
    if (!now) return;
    push(now);
    output(now->lft, v);
    v.pb(now->val);
    output(now->rt, v);
  }
  vector<int>get_arr() {
    vector<int>ret;
    output(root, ret);
    return ret;
  }
};
\end{lstlisting}
\subsection{LCA}
\begin{lstlisting}
/*Hey, What's up?*/

#include<bits/stdc++.h>
using namespace std;
#define pi acos(-1.0)
#define fastio ios_base::sync_with_stdio(false);cin.tie(NULL);cout.tie(NULL)
vector <int> v[100005], vc;
int x[200005][40], mp[100005], ml[100005], nd, pos[100005];

void build(int n) {
    for (int i = 0; i < n; i++)
        x[i][0] = vc[i];
    for (int a = 1, b = 1 ; a < n; a *= 2, b++)
        for (int i = 0; i < n - a; i++)
            x[i][b] = min(x[i][b - 1], x[i + a][b - 1]);
}
int query(int a, int b) {
    //if(a==b)return x[a][0];
    int c = 31 - __builtin_clz(b - a + 1); // log2(b - a + 1)
    int f = 1 << c;
    int d = x[a][c];
    int e = x[b - f + 1][c];
    return min(d, e);
}
void tour_de_euler(int p, int q) {
    vc.push_back(mp[p]);
    //nd++;
    if (!pos[mp[p]])pos[mp[p]] = nd;
    nd++;
    for (int e: v[p]) if(e != q) {
        tour_de_euler(e, p);
        vc.push_back(mp[p]); nd++;
    }
}
void dfs(int p, int q) {
    mp[p] = nd, ml[nd] = p;
    nd++;
    for (int e: v[p]) if (p != q)
        dfs(e, p);
}

int lca(int a, int b) {
    a = pos[mp[a]];
    b = pos[mp[b]];
    if (a > b)
        swap(a, b);
    int c = query(a, b);
    return ml[c];
}

int main()
{
    //fastio;
    int a = 0, b = 0, c, d, e, f = 0, l, g, m, n, k, i, j, t, p, q;
    cin >> n;
    for (i = 1; i < n; i++)
    {
        cin >> a >> b;
        v[a].push_back(b);
        v[b].push_back(a);
    }
    // nd = 1;
    // dfs(1, -1);
    // vc.push_back(6969696);
    // nd = 1;
    // tour_de_euler(1, -1);
    // l = vc.size();
    // build(l + 2);
    // cin >> q;
    // while (q--) {
    //     cin >> a >> b;
    //     cout << lca(a, b) << endl;
    // }

    return 0;
}
\end{lstlisting}
\subsection{Link Cut Tree}
\begin{lstlisting}
struct SplayTree {
  struct node {
    int ch[2] = {0, 0}, p = 0;
    ll self = 0, path = 0;
    ll sub = 0, extra = 0;
    bool rev = false;
  };
  vector<node> T;
  SplayTree(int n) : T(n + 1) {}
  void push(int x) {
    if (!x) return;
    int l = T[x].ch[0], r = T[x].ch[1];
    if (T[x].rev) {
      T[l].rev ^= true, T[r].rev ^= true;
      swap(T[x].ch[0], T[x].ch[1]);
      T[x].rev = false;
    }
  }
  void pull(int x) {
    int l = T[x].ch[0], r = T[x].ch[1];
    push(l), push(r);
    T[x].path = T[x].self + T[l].path + T[r].path;
    T[x].sub = T[x].self + T[x].extra + T[l].sub + T[r].sub;
  }
  void set(int parent, int child, int d) {
    T[parent].ch[d] = child;
    T[child].p = parent;
    pull(parent);
  }
  int dir(int x) {
    int parent = T[x].p;
    if (!parent) return -1;
    return (T[parent].ch[0] == x) ? 0 : (T[parent].ch[1] == x) ? 1 : -1;
  }
  void rotate(int x) {
    int parent = T[x].p, gparent = T[parent].p;
    int dx = dir(x), dp = dir(parent);
    set(parent, T[x].ch[!dx], dx);
    set(x, parent, !dx);
    if (~dp) set(gparent, x, dp);
    T[x].p = gparent;
  }
  void splay(int x) {
    push(x);
    while (~dir(x)) {
      int parent = T[x].p;
      int gparent = T[parent].p;
      push(gparent), push(parent), push(x);
      int dx = dir(x), dp = dir(parent);
      if (~dp) rotate(dx != dp ? x : parent);
      rotate(x);
    }
  }
};
struct LinkCut : SplayTree {
  LinkCut(int n) : SplayTree(n) {}
  void cut_right(int x) {
    splay(x);
    int r = T[x].ch[1];
    T[x].extra += T[r].sub;
    T[x].ch[1] = 0, pull(x);
  }
  int access(int x) {
    int u = x, v = 0;
    for (; u; v = u, u = T[u].p) {
      cut_right(u);
      T[u].extra -= T[v].sub;
      T[u].ch[1] = v, pull(u);
    }
    return splay(x), v;
  }
  void make_root(int x) {
    access(x);
    T[x].rev ^= true, push(x);
  }
  void link(int u, int v) {
    make_root(v), access(u);
    T[u].extra += T[v].sub;
    T[v].p = u, pull(u);
  }
  void cut(int u) {
    access(u);
    T[u].ch[0] = T[ T[u].ch[0] ].p = 0;
    pull(u);
  }
  void cut(int u, int v) {
    make_root(u), access(v);
    T[v].ch[0] = T[u].p = 0, pull(v);
  }
  int find_root(int u) {
    access(u), push(u);
    while (T[u].ch[0]) {
      u = T[u].ch[0], push(u);
    }
    return splay(u), u;
  }
  int lca(int u, int v) {
    if (u == v) return u;
    access(u);
    int ret = access(v);
    return T[u].p ? ret : 0;
  }
  // subtree query of u if v is the root
  ll subtree(int u, int v) {
    make_root(v), access(u);
    return T[u].self + T[u].extra;
  }
  ll path(int u, int v) {
    make_root(u), access(v);
    return T[v].path;
  }
  // point update
  void update(int u, ll val) {
    access(u);
    T[u].self = val, pull(u);
  }
};
\end{lstlisting}
\subsection{Merge Sort Tree}
\begin{lstlisting}
#include <ext/pb_ds/assoc_container.hpp>
#include <ext/pb_ds/tree_policy.hpp>
using namespace __gnu_pbds;

template<typename T> using ordered_set = tree<T, null_type, less<T>, rb_tree_tag, tree_order_statistics_node_update>;

ordered_set<pii> bst[MAX << 2];
void init(int n) {
  for (int i = 0; i <= 4 * n; i++) bst[i].clear();
}

void build(int now, int L, int R) {
  if (L == R) {
    bst[now].insert({arr[L], L});
    return;
  }
  for (int i = L; i <= R; i++) bst[now].insert({arr[i], i});

  int mid = (L + R) / 2;
  build(now << 1, L, mid);
  build((now << 1) | 1, mid + 1, R);
}

void update(int now, int L, int R, int idx, int ager_val, int val) {
  if (L == R) {
    bst[now].erase(bst[now].find({ager_val, idx}));
    bst[now].insert({val, idx});
    return;
  }

  int mid = (L + R) / 2;
  if (idx <= mid) update(now << 1, L, mid, idx, ager_val, val);
  else update((now << 1) | 1, mid + 1, R, idx, ager_val, val);
  bst[now].erase(bst[now].find({ager_val, idx}));
  bst[now].insert({val, idx});
}

ll query(int now, int L, int R, int i, int j, int val) {
  if (R < i || L > j) return 0;
  if (L >= i && R <= j) {
    int ret = bst[now].order_of_key({val, INT_MAX});
    return ret;
  }

  int mid = (L + R) / 2;
  return query(now << 1, L, mid, i, j, val) + query((now << 1) | 1, mid + 1, R, i, j, val);
}

\end{lstlisting}
\subsection{Mo on Tree}
\begin{lstlisting}
#include <bits/stdc++.h>
using namespace std;
const int N = 3e5 + 9;

//unique elements on the path from u to v
vector<int> g[N];
int st[N], en[N], T, par[N][20], dep[N], id[N * 2];
void dfs(int u, int p = 0) {
  st[u] = ++T;
  id[T] = u;
  dep[u] = dep[p] + 1;
  par[u][0] = p;
  for(int k = 1; k < 20; k++) par[u][k] = par[par[u][k - 1]][k - 1];
  for(auto v : g[u]) if(v != p) dfs(v, u);
  en[u] = ++T;
  id[T] = u;
}
int lca(int u, int v) {
  if(dep[u] < dep[v]) swap(u, v);
  for(int k = 19; k >= 0; k--) if(dep[par[u][k]] >= dep[v]) u = par[u][k];
  if(u == v) return u;
  for(int k = 19; k >= 0; k--) if(par[u][k] != par[v][k]) u = par[u][k], v = par[v][k];
  return par[u][0];
}

int cnt[N], a[N], ans;
inline void add(int u) {
  int x = a[u];
  if(cnt[x]++ == 0) ans++;
}
inline void rem(int u) {
  int x = a[u];
  if(--cnt[x] == 0) ans--;
}
bool vis[N];
inline void yo(int u) {
  if(!vis[u]) add(u);
  else rem(u);
  vis[u] ^= 1;
}

const int B = 320;
struct query {
  int l, r, id;
  bool operator < (const query &x) const {
    if(l / B == x.l / B) return r < x.r;
    return l / B < x.l / B;
  }
} Q[N];

int res[N];
int main() {
  ios_base::sync_with_stdio(0);
  cin.tie(0);

  int n, q;
  while(cin >> n >> q) {
    for(int i = 1; i <= n; i++) cin >> a[i];
    map<int, int> mp;
    for(int i = 1; i <= n; i++) {
      if(mp.find(a[i]) == mp.end()) mp[a[i]], mp[a[i]] = mp.size();
      a[i] = mp[a[i]];
    }
    for(int i = 1; i < n; i++) {
      int u, v;
      cin >> u >> v;
      g[u].push_back(v);
      g[v].push_back(u);
    }
    T = 0;
    dfs(1);
    for(int i = 1; i <= q; i++) {
      int u, v ;
      cin >> u >> v;
      if(st[u] > st[v]) swap(u, v);
      int lc = lca(u, v);
      if(lc == u) Q[i].l = st[u], Q[i].r = st[v];
      else Q[i].l = en[u], Q[i].r = st[v];
      Q[i].id = i;
    }
    sort(Q + 1, Q + q + 1);
    ans = 0;
    int l = 1, r = 0;
    for(int i = 1; i <= q; i++) {
      int L = Q[i].l, R = Q[i].r;
      if(R < l) {
        while (l > L) yo(id[--l]);
        while (l < L) yo(id[l++]);
        while (r < R) yo(id[++r]);
        while (r > R) yo(id[r--]);
      } else {
        while (r < R) yo(id[++r]);
        while (r > R) yo(id[r--]);
        while (l > L) yo(id[--l]);
        while (l < L) yo(id[l++]);
      }
      int u = id[l], v = id[r], lc = lca(u, v);
      if(lc != u && lc != v) yo(lc); //take care of the lca separately
      res[Q[i].id] = ans;
      if(lc != u && lc != v) yo(lc);
    }
    for(int i = 1; i <= q; i++) cout << res[i] << '\n';
    for(int i = 0; i <= n; i++) {
      g[i].clear();
      vis[i] = cnt[i] = 0;
      for(int k = 0; k < 20; k++) par[i][k] = 0;
    }
  }
  return 0;
}
\end{lstlisting}
\subsection{Mo's Algorithm}
\begin{lstlisting}
int vis[1000005];
int y[1000005];
int main()
{
    fastio;
    long long a=0,b=0,c,d,e,f=0,l,g,m,r,n,k,i,j,t,p,q;
    cin>>n;
    vector<long long>v;
    d=sqrt(1.0*n);
    vector<pair<long long,pair<long long,long long> > >x[d+2];
    //map<pair<long long,long long> ,long long>mp;
    v.push_back(-37);
    for(i=0;i<n;i++){
        cin>>a;
        v.push_back(a);
    }
    cin>>q;
    for(i=0;i<q;i++){
        cin>>a>>b;
        e=a/d;
        x[e].push_back({b,{a,i}});
    }
    for(i=0;i<=d+1;i++){
        sort(x[i].begin(),x[i].end());
    }
    for(i=0;i<=d;i++){
        memset(vis,0,sizeof(vis));
        l=i*d+1;
        r=i*d;
        p=x[i].size();
        f=0;
        for(j=0;j<p;j++){
            b=x[i][j].first;
            a=x[i][j].second.first;
            while(r<b){
                r++;
                vis[v[r]]++;
                if(vis[v[r]]==1)f++;
            }
            //cout<<l<<' '<<r<<'='<<f<<endl;
            if(l<a){
            while(l<a){
                vis[v[l]]--;
                if(vis[v[l]]==0)f--;
                l++;
            }}
            else if(l>a){
                while(l>a){
                    l--;
                    vis[v[l]]++;
                    if(vis[v[l]]==1)f++;
                }
            }
            //cout<<l<<' '<<r<<'='<<f<<endl;
            y[x[i][j].second.second]=f;
            //cout<<a<<' '<<b<<'='<<f<<endl;
        }
    }
    for(i=0;i<q;i++){
        cout<<y[i]<<'\n';
    }

    return 0;
}
\end{lstlisting}
\subsection{Persistent Segment Tree}
\begin{lstlisting}
struct node {
  int val, lft, rt;
  node(int val = 0, int lft = 0, int rt = 0) : val(val), lft(lft), rt(rt) {}
};
node nodes[30 * MAX]; ///take at least 2*n*log(n) nodes
int root[MAX], sz;
inline int update(int &now, int L, int R, int idx, int val) {
  if (L > idx || R < idx) return now;
  if (L == R) {
    ++sz;
    nodes[sz] = nodes[now];
    nodes[sz].val += val;
    return sz;
  }
  int mid = (L + R) >> 1;
  int ret = ++sz;
  if (idx <= mid) {
    if (!nodes[now].lft) nodes[now].lft = ++sz;
    nodes[ret].lft = update(nodes[now].lft, L, mid, idx, val);
    nodes[ret].rt = nodes[now].rt;
  } else {
    if (!nodes[now].rt) nodes[now].rt = ++sz;
    nodes[ret].rt = update(nodes[now].rt, mid + 1, R, idx, val);
    nodes[ret].lft = nodes[now].lft;
  }
  nodes[ret].val = nodes[ nodes[ret].lft ].val + nodes[ nodes[ret].rt ].val;
  return ret;
}
inline int query(int &now, int L, int R, int i, int j) {
  if (L > j || R < i) return 0;
  if (L >= i && R <= j) return nodes[now].val;
  int mid = (L + R) >> 1;
  return query(nodes[now].lft, L, mid, i, j) + query(nodes[now].rt, mid + 1, R, i, j);
}
/// in main(make segtree for every prefix)
root[0] = 0;
for (int i = 1; i <= n; i++) root[i] = update(root[i - 1], 1, n, p[i], 1);
\end{lstlisting}
\subsection{Sparse Table}
\begin{lstlisting}
const int maxn = (1<<20)+5 ;
int logs[maxn] = {0};

void compute_logs(){
    logs[1] = 0;
    for(int i=2;i<(1<<20);i++){
        logs[i] = logs[i/2]+1;
    }
}

class Sparse_Table
{
    public:
        vector <vector<LL>> table; 
        function < LL(LL,LL) > func;
        LL identity;

    Sparse_Table(vector <LL> &v, function <LL(LL,LL)> _func, LL id){
        if(logs[2] != 1) compute_logs();
        int sz = v.size();
        table.assign(sz,vector <LL>(logs[sz]+1) );
        func = _func, identity = id;

        for(int j=0;j<=logs[sz];j++){
            for(int i=0;i+(1<<j)<=sz;i++){
                if(j==0) table[i][j] = func(v[i],id);    // base case, when only 1 element in range
                else table[i][j] = func(table[i][j-1], table[i + (1<<(j-1))][j-1] );
            }
        }
    }
    // when intersection of two ranges wont be a problem like min, gcd,max
    LL query(int l, int r){
        assert(r>=l);
        int pow = logs[r-l+1];
        return func(table[l][pow], table[r- (1<<pow) + 1][pow]);
    }
    // other cases like sum
    LL Query(int l,int r){
        if(l>r) return identity; // handle basecase
        int pow = logs[r - l + 1];
        return func(table[l][pow], Query(l+(1<<pow), r));
    }
};\end{lstlisting}
\subsection{Treap}
\begin{lstlisting}
mt19937 rnd(chrono::steady_clock::now().time_since_epoch().count());
typedef struct node* pnode;
struct node {
  int prior, val, sz;
  ll sum;
  node *lft, *rt;
  node(int val = 0, node *lft = NULL, node *rt = NULL) : 
    lft(lft), rt(rt), prior(rnd()), val(val), sz(1), sum(0) {}
};
struct treap {
  pnode root;
  treap() {
    root = NULL;
  }
  int get_sz(pnode now) {
    return now ? now->sz : 0;
  }
  void update_sz(pnode now) {
    if (!now) return;
    now->sz = 1 + get_sz(now->lft) + get_sz(now->rt);
  }
  ll get(pnode now) {
    return now ? now->sum : 0;
  }
  void push(pnode now) {}
  void combine(pnode now) {
    if (!now) return;
    now->sum = now->val + get(now->lft) + get(now->rt);
  }
  pnode unite(pnode lft, pnode rt) {
    if (!lft || !rt) return lft ? lft : rt;
    // push(lft), push(rt); this not tested
    if (lft->prior < rt->prior) swap(lft, rt);
    pnode l, r;
    split(rt, l, r, lft->val);
    lft->lft = unite(lft->lft, l), update_sz(lft);
    lft->rt = unite(lft->rt, r), update_sz(lft);
    // combine(lft); this not tested
    return lft;
  }
  ///value < val goes to left, value >= val goes to right
  void split(pnode now, pnode &lft, pnode &rt, int val, int add = 0) {
    push(now);
    if (!now) return void(lft = rt = NULL);
    if (now->val < val) split(now->rt, now->rt, rt, val), lft = now;
    else split(now->lft, lft, now->lft, val), rt = now;
    update_sz(now), combine(now);
  }
  void merge(pnode &now, pnode lft, pnode rt) {
    push(lft), push(rt);
    if (!lft || !rt) now = lft ? lft : rt;
    else if (lft->prior > rt->prior) merge(lft->rt, lft->rt, rt), now = lft;
    else merge(rt->lft, lft, rt->lft), now = rt;
    update_sz(now), combine(now);
  }
  void insert(pnode &now, pnode notun) {
    if (!now) return void(now = notun);
    push(now);
    if (notun->prior > now->prior) split(now, notun->lft, notun->rt, notun->val), now = notun;
    else insert(notun->val < now->val ? now->lft : now->rt, notun);
    update_sz(now), combine(now);
  }
  void erase(pnode &now, int val) {
    push(now);
    if (now->val == val) {
      pnode temp = now;
      merge(now, now->lft, now->rt);
      delete(temp);
    } else erase(val < now->val ? now->lft : now->rt, val);
    update_sz(now), combine(now);
  }
  int get_idx(pnode &now, int val) {
    if (!now) return INT_MIN;
    else if (now->val == val) return 1 + get_sz(now->lft);
    else if (val < now->val) return get_idx(now->lft, val);
    else return (1 + get_sz(now->lft) + get_idx(now->rt, val));
  }
  int find_kth(pnode &now, int k) {
    if (k < 1 || k > get_sz(now)) return -1;
    if (get_sz(now->lft) + 1 == k) return now->val;
    if (k <= get_sz(now->lft)) return find_kth(now->lft, k);
    return find_kth(now->rt, k - get_sz(now->lft) - 1);
  }
  ll prefix_sum(pnode &now, int k) {
    if (k < 1 || k > get_sz(now)) return -inf;
    if (get_sz(now->lft) + 1 == k) return get(now->lft) + now->val;
    if (k <= get_sz(now->lft)) return prefix_sum(now->lft, k);
    return get(now->lft) + now->val + prefix_sum(now->rt, k - get_sz(now->lft) - 1);
  }
  pnode get_rng(int l, int r) { ///gets all l <= values <= r
    pnode lft, rt, mid;
    split(root, lft, mid, l);
    split(mid, mid, rt, r + 1);
    merge(root, lft, rt);
    return mid;
  }
  void output(pnode now, vector<int>&v) {
    if (!now) return;
    output(now->lft, v);
    v.pb(now->val);
    output(now->rt, v);
  }
  vector<int>get_arr() {
    vector<int>ret;
    output(root, ret);
    return ret;
  }
};
\end{lstlisting}
\subsection{Trie}
\begin{lstlisting}
int trie[30 * 100000 + 5][2];
int mark[30 * 100000 + 5];
int node = 1;
void add(int n) {
  int now = 1;
  for (int i = 27; i >= 0; i--) {
    int d = (bool)(n & (1 << i));
    if (!trie[now][d]) trie[now][d] = ++node;
    now = trie[now][d];
    mark[now]++;
  }
}
void del(int n) {
  int now = 1;
  deque<int>v;
  for (int i = 27; i >= 0; i--) {
    int d = (bool)(n & (1 << i));
    if (trie[now][d]) {
      v.push_front(now);
      now = trie[now][d];
      mark[now]--;
    }
  }
  v.push_front(now);
  for (int i = 1; i < v.size(); i++) {
    if (!mark[v[i - 1]]) {
      if (trie[v[i]][0] == v[i - 1]) trie[v[i]][0] = 0;
      if (trie[v[i]][1] == v[i - 1]) trie[v[i]][1] = 0;
    }
  }
}
\end{lstlisting}
\section{Geometry}
\subsection{2D Point}
\begin{lstlisting}
typedef double Tf;
typedef Tf Ti;      /// use long long for exactness
const Tf PI = acos(-1), EPS = 1e-9;
int dcmp(Tf x) { return abs(x) < EPS ? 0 : (x<0 ? -1 : 1);}

struct Point {
  Ti x, y;
  Point(Ti x = 0, Ti y = 0) : x(x), y(y) {}

  Point operator + (const Point& u) const { return Point(x + u.x, y + u.y); }
  Point operator - (const Point& u) const { return Point(x - u.x, y - u.y); }
  Point operator * (const long long u) const { return Point(x * u, y * u); }
  Point operator * (const Tf u) const { return Point(x * u, y * u); }
  Point operator / (const Tf u) const { return Point(x / u, y / u); }

  bool operator == (const Point& u) const { return dcmp(x - u.x) == 0 && dcmp(y - u.y) == 0; }
  bool operator != (const Point& u) const { return !(*this == u); }
  bool operator < (const Point& u) const { return dcmp(x - u.x) < 0 || (dcmp(x - u.x) == 0 && dcmp(y - u.y) < 0); }
};

Ti dot(Point a, Point b) { return a.x * b.x + a.y * b.y; }
Ti cross(Point a, Point b) { return a.x * b.y - a.y * b.x; }
Tf length(Point a) { return sqrt(dot(a, a)); }
Ti sqLength(Point a) { return dot(a, a); }
Tf distance(Point a, Point b) {return length(a-b);}
Tf angle(Point u) { return atan2(u.y, u.x); }

// returns angle between oa, ob in (-PI, PI]
Tf angleBetween(Point a, Point b) {
  double ans = angle(b) - angle(a);
  return ans <= -PI ? ans + 2*PI : (ans > PI ? ans - 2*PI : ans);
}

// Rotate a ccw by rad radians
Point rotate(Point a, Tf rad) {
  static_assert(is_same<Tf, Ti>::value);
  return Point(a.x * cos(rad) - a.y * sin(rad), a.x * sin(rad) + a.y * cos(rad));
}

// rotate a ccw by angle th with cos(th) = co && sin(th) = si
Point rotatePrecise(Point a, Tf co, Tf si) {
  static_assert(is_same<Tf, Ti>::value);
  return Point(a.x * co - a.y * si, a.y * co + a.x * si);
}

Point rotate90(Point a) { return Point(-a.y, a.x); }

// scales vector a by s such that length of a becomes s
Point scale(Point a, Tf s) {
  static_assert(is_same<Tf, Ti>::value);
  return a / length(a) * s;
}

// returns an unit vector perpendicular to vector a
Point normal(Point a) {
  static_assert(is_same<Tf, Ti>::value);
  Tf l = length(a);
  return Point(-a.y / l, a.x / l);
}

// returns 1 if c is left of ab, 0 if on ab && -1 if right of ab
int orient(Point a, Point b, Point c) {
  return dcmp(cross(b - a, c - a));
}

class polarComp {
    point O, dir;
    bool half(point p) {
        return dcmp(dir & p) < 0 || (dcmp(dir & p) == 0 && dcmp(dir ^ p) > 0);
    }
    public:
    polarComp(point O = point(0, 0), point dir = point(1, 0))
        : O(O), dir(dir) {}
  
    bool operator() (point p, point q) {
        return make_tuple(half(p), 0) < make_tuple(half(q), (p & q));
    }
}; // given a pivot point and an initial direction, sorts by Angle with the given direction

struct Segment {
  Point a, b;
  Segment(Point aa, Point bb) : a(aa), b(bb) {}
};
typedef Segment Line;

struct Circle {
    Point o;
    Tf r;
    Circle(Point o = Point(0, 0), Tf r = 0) : o(o), r(r) {}

    // returns true if point p is in || on the circle
    bool contains(Point p) {
      return dcmp(sqLength(p - o) - r * r) <= 0;
    }

    // returns a point on the circle rad radians away from +X CCW
    Point point(Tf rad) {
      static_assert(is_same<Tf, Ti>::value);
      return Point(o.x + cos(rad) * r, o.y + sin(rad) * r);
    }

    // area of a circular sector with central angle rad
    Tf area(Tf rad = PI + PI) { return rad * r * r / 2; }

    // area of the circular sector cut by a chord with central angle alpha
    Tf sector(Tf alpha) { return r * r * 0.5 * (alpha - sin(alpha)); }
};

\end{lstlisting}
\subsection{Circle Algorithms}
\begin{lstlisting}
// Extremely inaccurate for finding near touches
// compute intersection of line l with circle c
// The intersections are given in order of the ray (l.a, l.b)
vector<Point> circleLineIntersection(Circle c, Line l) {
    static_assert(is_same<Tf, Ti>::value);
    vector<Point> ret;
    Point b = l.b - l.a, a = l.a - c.o;

    Tf A = dot(b, b), B = dot(a, b);
    Tf C = dot(a, a) - c.r * c.r, D = B * B - A * C;
    if (D < -EPS) return ret;

    ret.push_back(l.a + b * (-B - sqrt(D + EPS)) / A);
    if (D > EPS) ret.push_back(l.a + b * (-B + sqrt(D)) / A);
    return ret;
}

// signed area of intersection of circle(c.o, c.r) &&
// triangle(c.o, s.a, s.b) [cross(a-o, b-o)/2]
Tf circleTriangleIntersectionArea(Circle c, Segment s) {
    Tf OA = length(c.o - s.a);
    Tf OB = length(c.o - s.b);

    // sector
    if (dcmp(distancePointSegment(c.o, s) - c.r) >= 0)
        return angleBetween(s.a - c.o, s.b - c.o) * (c.r * c.r) / 2.0;

    // triangle
    if (dcmp(OA - c.r) <= 0 && dcmp(OB - c.r) <= 0)
        return cross(c.o - s.b, s.a - s.b) / 2.0;

    // three part: (A, a) (a, b) (b, B)
    vector<Point> Sect = circleLineIntersection(c, s);
    return circleTriangleIntersectionArea(c, Segment(s.a, Sect[0])) +
           circleTriangleIntersectionArea(c, Segment(Sect[0], Sect[1])) +
           circleTriangleIntersectionArea(c, Segment(Sect[1], s.b));
}

// area of intersecion of circle(c.o, c.r) && simple polyson(p[])
// Tested : https://codeforces.com/gym/100204/problem/F - Little Mammoth
Tf circlePolyIntersectionArea(Circle c, Polygon p) {
    Tf res = 0;
    int n = p.size();
    for (int i = 0; i < n; ++i)
        res += circleTriangleIntersectionArea(c, Segment(p[i], p[(i + 1) % n]));
    return abs(res);
}

// locates circle c2 relative to c1
// interior       (d < R - r)     ----> -2
// interior tangents (d = R - r)     ----> -1
// concentric    (d = 0)
// secants       (R - r < d < R + r) ---->  0
// exterior tangents (d = R + r)     ---->  1
// exterior       (d > R + r)     ---->  2
int circleCirclePosition(Circle c1, Circle c2) {
    Tf d = length(c1.o - c2.o);
    int in = dcmp(d - abs(c1.r - c2.r)), ex = dcmp(d - (c1.r + c2.r));
    return in < 0 ? -2 : in == 0 ? -1 : ex == 0 ? 1 : ex > 0 ? 2 : 0;
}

// compute the intersection points between two circles c1 && c2
vector<Point> circleCircleIntersection(Circle c1, Circle c2) {
    static_assert(is_same<Tf, Ti>::value);
    vector<Point> ret;
    Tf d = length(c1.o - c2.o);
    if (dcmp(d) == 0) return ret;
    if (dcmp(c1.r + c2.r - d) < 0) return ret;
    if (dcmp(abs(c1.r - c2.r) - d) > 0) return ret;

    Point v = c2.o - c1.o;
    Tf co = (c1.r * c1.r + sqLength(v) - c2.r * c2.r) / (2 * c1.r * length(v));
    Tf si = sqrt(abs(1.0 - co * co));
    Point p1 = scale(rotatePrecise(v, co, -si), c1.r) + c1.o;
    Point p2 = scale(rotatePrecise(v, co, si), c1.r) + c1.o;

    ret.push_back(p1);
    if (p1 != p2) ret.push_back(p2);
    return ret;
}

// intersection area between two circles c1, c2
Tf circleCircleIntersectionArea(Circle c1, Circle c2) {
    Point AB = c2.o - c1.o;
    Tf d = length(AB);
    if (d >= c1.r + c2.r) return 0;
    if (d + c1.r <= c2.r) return PI * c1.r * c1.r;
    if (d + c2.r <= c1.r) return PI * c2.r * c2.r;

    Tf alpha1 = acos((c1.r * c1.r + d * d - c2.r * c2.r) / (2.0 * c1.r * d));
    Tf alpha2 = acos((c2.r * c2.r + d * d - c1.r * c1.r) / (2.0 * c2.r * d));
    return c1.sector(2 * alpha1) + c2.sector(2 * alpha2);
}

// returns tangents from a point p to circle c
vector<Point> pointCircleTangents(Point p, Circle c) {
    static_assert(is_same<Tf, Ti>::value);

    vector<Point> ret;
    Point u = c.o - p;
    Tf d = length(u);
    if (d < c.r)
        ;
    else if (dcmp(d - c.r) == 0) {
        ret = {rotate(u, PI / 2)};
    } else {
        Tf ang = asin(c.r / d);
        ret = {rotate(u, -ang), rotate(u, ang)};
    }
    return ret;
}

// returns the points on tangents that touches the circle
vector<Point> pointCircleTangencyPoints(Point p, Circle c) {
    static_assert(is_same<Tf, Ti>::value);

    Point u = p - c.o;
    Tf d = length(u);
    if (d < c.r)
        return {};
    else if (dcmp(d - c.r) == 0)
        return {c.o + u};
    else {
        Tf ang = acos(c.r / d);
        u = u / length(u) * c.r;
        return {c.o + rotate(u, -ang), c.o + rotate(u, ang)};
    }
}

// for two circles c1 && c2, returns two list of points a && b
// such that a[i] is on c1 && b[i] is c2 && for every i
// Line(a[i], b[i]) is a tangent to both circles
// CAUTION: a[i] = b[i] in case they touch | -1 for c1 = c2
int circleCircleTangencyPoints(Circle c1, Circle c2, vector<Point> &a,
                               vector<Point> &b) {
    a.clear(), b.clear();
    int cnt = 0;
    if (dcmp(c1.r - c2.r) < 0) {
        swap(c1, c2);
        swap(a, b);
    }
    Tf d2 = sqLength(c1.o - c2.o);
    Tf rdif = c1.r - c2.r, rsum = c1.r + c2.r;
    if (dcmp(d2 - rdif * rdif) < 0) return 0;
    if (dcmp(d2) == 0 && dcmp(c1.r - c2.r) == 0) return -1;

    Tf base = angle(c2.o - c1.o);
    if (dcmp(d2 - rdif * rdif) == 0) {
        a.push_back(c1.point(base));
        b.push_back(c2.point(base));
        cnt++;
        return cnt;
    }

    Tf ang = acos((c1.r - c2.r) / sqrt(d2));
    a.push_back(c1.point(base + ang));
    b.push_back(c2.point(base + ang));
    cnt++;
    a.push_back(c1.point(base - ang));
    b.push_back(c2.point(base - ang));
    cnt++;

    if (dcmp(d2 - rsum * rsum) == 0) {
        a.push_back(c1.point(base));
        b.push_back(c2.point(PI + base));
        cnt++;
    } else if (dcmp(d2 - rsum * rsum) > 0) {
        Tf ang = acos((c1.r + c2.r) / sqrt(d2));
        a.push_back(c1.point(base + ang));
        b.push_back(c2.point(PI + base + ang));
        cnt++;
        a.push_back(c1.point(base - ang));
        b.push_back(c2.point(PI + base - ang));
        cnt++;
    }
    return cnt;
}
\end{lstlisting}
\subsection{Closest Pair of Points}
\begin{lstlisting}
LL ClosestPair(vector<pii> pts) {
    int n = pts.size();
    sort(all(pts));
    set<pii> s;

    LL best_dist = 1e18;
    int j = 0;
    for (int i = 0; i < n; ++i) {
        int d = ceil(sqrt(best_dist));
        while (pts[i].ff - pts[j].ff >= best_dist) {
            s.erase({pts[j].ss, pts[j].ff});
            j += 1;
        }

        auto it1 = s.lower_bound({pts[i].ss - d, pts[i].ff});
        auto it2 = s.upper_bound({pts[i].ss + d, pts[i].ff});

        for (auto it = it1; it != it2; ++it) {
            int dx = pts[i].ff - it->ss;
            int dy = pts[i].ss - it->ff;
            best_dist = min(best_dist, 1LL * dx * dx + 1LL * dy * dy);
        }
        s.insert({pts[i].ss, pts[i].ff});
    }
    return best_dist;
}
\end{lstlisting}
\subsection{Linear}
\begin{lstlisting}
// returns true if point p is on segment s
bool onSegment(Point p, Segment s) {
    return dcmp(cross(s.a - p, s.b - p)) == 0 &&
           dcmp(dot(s.a - p, s.b - p)) <= 0;
}

// returns true if segment p && q touch or intersect
bool segmentsIntersect(Segment p, Segment q) {
    if (onSegment(p.a, q) || onSegment(p.b, q)) return true;
    if (onSegment(q.a, p) || onSegment(q.b, p)) return true;

    Ti c1 = cross(p.b - p.a, q.a - p.a);
    Ti c2 = cross(p.b - p.a, q.b - p.a);
    Ti c3 = cross(q.b - q.a, p.a - q.a);
    Ti c4 = cross(q.b - q.a, p.b - q.a);
    return dcmp(c1) * dcmp(c2) < 0 && dcmp(c3) * dcmp(c4) < 0;
}

bool linesParallel(Line p, Line q) {
    return dcmp(cross(p.b - p.a, q.b - q.a)) == 0;
}

// lines are represented as a ray from a point: (point, vector)
// returns false if two lines (p, v) && (q, w) are parallel or collinear
// true otherwise, intersection point is stored at o via reference
bool lineLineIntersection(Point p, Point v, Point q, Point w, Point& o) {
    static_assert(is_same<Tf, Ti>::value);
    if (dcmp(cross(v, w)) == 0) return false;
    Point u = p - q;
    o = p + v * (cross(w, u) / cross(v, w));
    return true;
}

// returns false if two lines p && q are parallel or collinear
// true otherwise, intersection point is stored at o via reference
bool lineLineIntersection(Line p, Line q, Point& o) {
    return lineLineIntersection(p.a, p.b - p.a, q.a, q.b - q.a, o);
}

// returns the distance from point a to line l
Tf distancePointLine(Point p, Line l) {
    return abs(cross(l.b - l.a, p - l.a) / length(l.b - l.a));
}

// returns the shortest distance from point a to segment s
Tf distancePointSegment(Point p, Segment s) {
    if (s.a == s.b) return length(p - s.a);
    Point v1 = s.b - s.a, v2 = p - s.a, v3 = p - s.b;
    if (dcmp(dot(v1, v2)) < 0)
        return length(v2);
    else if (dcmp(dot(v1, v3)) > 0)
        return length(v3);
    else return abs(cross(v1, v2) / length(v1));
}

// returns the shortest distance from segment p to segment q
Tf distanceSegmentSegment(Segment p, Segment q) {
    if (segmentsIntersect(p, q)) return 0;
    Tf ans = distancePointSegment(p.a, q);
    ans = min(ans, distancePointSegment(p.b, q));
    ans = min(ans, distancePointSegment(q.a, p));
    ans = min(ans, distancePointSegment(q.b, p));
    return ans;
}

// returns the projection of point p on line l
Point projectPointLine(Point p, Line l) {
    static_assert(is_same<Tf, Ti>::value);
    Point v = l.b - l.a;
    return l.a + v * ((Tf)dot(v, p - l.a) / dot(v, v));
}
\end{lstlisting}
\subsection{Pair of Intersecting segments using Line Sweep}
\begin{lstlisting}
/*
Checking for the intersection of two segments is carried out by the intersect ()
function, using an algorithm based on the oriented area of the triangle.

The queue of segments is the global variable s, a set<event>. Iterators that
specify the position of each segment in the queue (for convenient removal of
segments from the queue) are stored in the global array where.

Two auxiliary functions prev() and next() are also introduced, which return
iterators to the previous and next elements (or end(), if one does not exist).
*/
Tf yvalSegment(const Line &s, Tf x) {
    if (dcmp(s.a.x - s.b.x) == 0) return s.a.y;
    return s.a.y + (s.b.y - s.a.y) * (x - s.a.x) / (s.b.x - s.a.x);
}

struct SegCompare {
    bool operator()(const Segment &p, const Segment &q) const {
        Tf x = max(min(p.a.x, p.b.x), min(q.a.x, q.b.x));
        return dcmp(yvalSegment(p, x) - yvalSegment(q, x)) < 0;
    }
};

multiset<Segment, SegCompare> st;
typedef multiset<Segment, SegCompare>::iterator iter;

iter prev(iter it) { return it == st.begin() ? st.end() : --it; }

iter next(iter it) { return it == st.end() ? st.end() : ++it; }

struct Event {
    Tf x;
    int tp, id;
    Event(Ti x, int tp, int id) : x(x), tp(tp), id(id) {}
    bool operator<(const Event &p) const {
        if (dcmp(x - p.x)) return x < p.x;
        return tp > p.tp;
    }
};

bool anyIntersection(const vector<Segment> &v) {
    using Linear::segmentsIntersect;
    vector<Event> ev;
    for (int i = 0; i < (int)v.size(); ++i) {
        ev.push_back(Event(min(v[i].a.x, v[i].b.x), +1, i));
        ev.push_back(Event(max(v[i].a.x, v[i].b.x), -1, i));
    }
    sort(ev.begin(), ev.end());

    st.clear();
    vector<iter> where(v.size());
    for (auto &cur : ev) {
        int id = cur.id;
        if (cur.tp == 1) {
            iter nxt = st.lower_bound(v[id]);
            iter pre = prev(nxt);
            if (pre != st.end() && segmentsIntersect(*pre, v[id])) return true;
            if (nxt != st.end() && segmentsIntersect(*nxt, v[id])) return true;
            where[id] = st.insert(nxt, v[id]);
        } else {
            iter nxt = next(where[id]);
            iter pre = prev(where[id]);
            if (pre != st.end() && nxt != st.end() &&
                segmentsIntersect(*pre, *nxt))
                return true;
            st.erase(where[id]);
        }
    }
    return false;
}
\end{lstlisting}
\subsection{Polygons}
\begin{lstlisting}
// returns the signed area of polygon p of n vertices
using Polygon = vector<Point>;
Tf signedPolygonArea(Polygon p) {
    Tf ret = 0;
    for (int i = 0; i < (int)p.size() - 1; i++)
        ret += cross(p[i] - p[0], p[i + 1] - p[0]);
    return ret / 2;
}
// given a polygon p of n vertices, generates the convex hull in ch
// in CCW && returns the number of vertices in the convex hull
int convexHull(Polygon p, Polygon &ch) {
    sort(p.begin(), p.end());
    int n = p.size();
    ch.resize(n + n);
    int m = 0;  // preparing lower hull
    for (int i = 0; i < n; i++) {
        while (m > 1 &&
               dcmp(cross(ch[m - 1] - ch[m - 2], p[i] - ch[m - 1])) <= 0)
            m--;
        ch[m++] = p[i];
    }
    int k = m;  // preparing upper hull
    for (int i = n - 2; i >= 0; i--) {
        while (m > k &&
               dcmp(cross(ch[m - 1] - ch[m - 2], p[i] - ch[m - 2])) <= 0)
            m--;
        ch[m++] = p[i];
    }
    if (n > 1) m--;
    ch.resize(m);
    return m;
}

// for a point o and polygon p returns:
//   -1 if o is strictly inside p
//  0 if o is on a segment of p
//  1 if o is strictly outside p
// computes via winding numbers
int pointInPolygon(Point o, Polygon p) {
    int wn = 0, n = p.size();
    for (int i = 0; i < n; i++) {
        int j = (i + 1) % n;
        if (onSegment(o, Segment(p[i], p[j])) || o == p[i]) return 0;
        int k = dcmp(cross(p[j] - p[i], o - p[i]));
        int d1 = dcmp(p[i].y - o.y);
        int d2 = dcmp(p[j].y - o.y);
        if (k > 0 && d1 <= 0 && d2 > 0) wn++;
        if (k < 0 && d2 <= 0 && d1 > 0) wn--;
    }
    return wn ? -1 : 1;
}

// returns the longest line segment of l that is inside or on the
// simply polygon p. O(n lg n). TESTED: TIMUS 1955
Tf longestSegInPoly(Line l, const Polygon &p) {
    int n = p.size();
    vector<pair<Tf, int>> ev;
    for (int i = 0; i < n; ++i) {
        Point a = p[i], b = p[(i + 1) % n], z = p[(i - 1 + n) % n];
        int ora = orient(l.a, l.b, a), orb = orient(l.a, l.b, b),
            orz = orient(l.a, l.b, z);
        if (!ora) {
            Tf d = dot(a - l.a, l.b - l.a);
            if (orz && orb) {
                if (orz != orb) ev.emplace_back(d, 0);
            } else if (orz)
                ev.emplace_back(d, orz);
            else if (orb)
                ev.emplace_back(d, orb);
        } else if (ora == -orb) {
            Point ins;
            lineLineIntersection(l, Line(a, b), ins);
            ev.emplace_back(dot(ins - l.a, l.b - l.a), 0);
        }
    }
    sort(ev.begin(), ev.end());

    Tf ret = 0, cur = 0, pre = 0;
    bool active = false;
    int sign = 0;
    for (auto &qq : ev) {
        int tp = qq.second;
        Tf d = qq.first;
        if (sign) {
            cur += d - pre;
            ret = max(ret, cur);
            if (tp != sign) active = !active;
            sign = 0;
        } else {
            if (active) cur += d - pre, ret = max(ret, cur);
            if (tp == 0)
                active = !active;
            else
                sign = tp;
        }
        pre = d;
        if (!active) cur = 0;
    }

    ret /= length(l.b - l.a);
    return ret;
}

// EVERYTHING AFTER THIS ONLY FOR CONVEX POLYGON
/// Tested on Kattis::fenceortho
void rotatingCalipersGetRectangle(Point *p, int n, Tf &area, Tf &perimeter) {
    static_assert(is_same<Tf, Ti>::value);
    p[n] = p[0];
    int l = 1, r = 1, j = 1;
    area = perimeter = 1e100;

    for (int i = 0; i < n; i++) {
        Point v = (p[i + 1] - p[i]) / length(p[i + 1] - p[i]);
        while (dcmp(dot(v, p[r % n] - p[i]) - dot(v, p[(r + 1) % n] - p[i])) <
               0)
            r++;
        while (j < r || dcmp(cross(v, p[j % n] - p[i]) -
                             cross(v, p[(j + 1) % n] - p[i])) < 0)
            j++;
        while (l < j || dcmp(dot(v, p[l % n] - p[i]) -
                             dot(v, p[(l + 1) % n] - p[i])) > 0)
            l++;
        Tf w = dot(v, p[r % n] - p[i]) - dot(v, p[l % n] - p[i]);
        Tf h = distancePointLine(p[j % n], Line(p[i], p[i + 1]));
        area = min(area, w * h);
        perimeter = min(perimeter, 2 * w + 2 * h);
    }
}

// returns the left side of polygon u after cutting it by ray a->b
Polygon cutPolygon(Polygon u, Point a, Point b) {
    Polygon ret;
    int n = u.size();
    for (int i = 0; i < n; i++) {
        Point c = u[i], d = u[(i + 1) % n];
        if (dcmp(cross(b - a, c - a)) >= 0) ret.push_back(c);
        if (dcmp(cross(b - a, d - c)) != 0) {
            Point t;
            lineLineIntersection(a, b - a, c, d - c, t);
            if (onSegment(t, Segment(c, d))) ret.push_back(t);
        }
    }
    return ret;
}

// returns true if point p is in or on triangle abc
bool pointInTriangle(Point a, Point b, Point c, Point p) {
    return dcmp(cross(b - a, p - a)) >= 0 && dcmp(cross(c - b, p - b)) >= 0 &&
           dcmp(cross(a - c, p - c)) >= 0;
}

// Tested : https://www.spoj.com/problems/INOROUT
// pt must be in ccw order with no three collinear points
// returns inside = -1, on = 0, outside = 1
int pointInConvexPolygon(const Polygon &pt, Point p) {
    int n = pt.size();
    assert(n >= 3);

    int lo = 1, hi = n - 1;
    while (hi - lo > 1) {
        int mid = (lo + hi) / 2;
        if (dcmp(cross(pt[mid] - pt[0], p - pt[0])) > 0)
            lo = mid;
        else
            hi = mid;
    }

    bool in = pointInTriangle(pt[0], pt[lo], pt[hi], p);
    if (!in) return 1;

    if (dcmp(cross(pt[lo] - pt[lo - 1], p - pt[lo - 1])) == 0) return 0;
    if (dcmp(cross(pt[hi] - pt[lo], p - pt[lo])) == 0) return 0;
    if (dcmp(cross(pt[hi] - pt[(hi + 1) % n], p - pt[(hi + 1) % n])) == 0)
        return 0;
    return -1;
}

// Extreme Point for a direction is the farthest point in that direction
// O'Rourke, page 270, http://crtl-i.com/PDF/comp_c.pdf
// also https://codeforces.com/blog/entry/48868
// poly is a convex polygon, sorted in CCW, doesn't contain redundant points
// u is the direction for extremeness
int extremePoint(const Polygon &poly, Point u = Point(0, 1)) {
    int n = (int)poly.size();
    int a = 0, b = n;
    while (b - a > 1) {
        int c = (a + b) / 2;
        if (dcmp(dot(poly[c] - poly[(c + 1) % n], u)) >= 0 &&
            dcmp(dot(poly[c] - poly[(c - 1 + n) % n], u)) >= 0) {
            return c;
        }

        bool a_up = dcmp(dot(poly[(a + 1) % n] - poly[a], u)) >= 0;
        bool c_up = dcmp(dot(poly[(c + 1) % n] - poly[c], u)) >= 0;
        bool a_above_c = dcmp(dot(poly[a] - poly[c], u)) > 0;

        if (a_up && !c_up)
            b = c;
        else if (!a_up && c_up)
            a = c;
        else if (a_up && c_up) {
            if (a_above_c)
                b = c;
            else
                a = c;
        } else {
            if (!a_above_c)
                b = c;
            else
                a = c;
        }
    }

    if (dcmp(dot(poly[a] - poly[(a + 1) % n], u)) > 0 &&
        dcmp(dot(poly[a] - poly[(a - 1 + n) % n], u)) > 0)
        return a;
    return b % n;
}

// For a convex polygon p and a line l, returns a list of segments
// of p that are touch or intersect line l.
// the i'th segment is considered (p[i], p[(i + 1) modulo |p|])
// #1 If a segment is collinear with the line, only that is returned
// #2 Else if l goes through i'th point, the i'th segment is added
// If there are 2 or more such collinear segments for #1,
// any of them (only one, not all) should be returned (not tested)
// Complexity: O(lg |p|)
vector<int> lineConvexPolyIntersection(const Polygon &p, Line l) {
    assert((int)p.size() >= 3);
    assert(l.a != l.b);

    int n = p.size();
    vector<int> ret;

    Point v = l.b - l.a;
    int lf = extremePoint(p, rotate90(v));
    int rt = extremePoint(p, rotate90(v) * Ti(-1));
    int olf = orient(l.a, l.b, p[lf]);
    int ort = orient(l.a, l.b, p[rt]);

    if (!olf || !ort) {
        int idx = (!olf ? lf : rt);
        if (orient(l.a, l.b, p[(idx - 1 + n) % n]) == 0)
            ret.push_back((idx - 1 + n) % n);
        else
            ret.push_back(idx);
        return ret;
    }
    if (olf == ort) return ret;

    for (int i = 0; i < 2; ++i) {
        int lo = i ? rt : lf;
        int hi = i ? lf : rt;
        int olo = i ? ort : olf;

        while (true) {
            int gap = (hi - lo + n) % n;
            if (gap < 2) break;

            int mid = (lo + gap / 2) % n;
            int omid = orient(l.a, l.b, p[mid]);
            if (!omid) {
                lo = mid;
                break;
            }
            if (omid == olo)
                lo = mid;
            else
                hi = mid;
        }
        ret.push_back(lo);
    }
    return ret;
}

// Tested : https://toph.co/p/cover-the-points
// Calculate [ACW, CW] tangent pair from an external point
constexpr int CW = -1, ACW = 1;
bool isGood(Point u, Point v, Point Q, int dir) {
    return orient(Q, u, v) != -dir;
}
Point better(Point u, Point v, Point Q, int dir) {
    return orient(Q, u, v) == dir ? u : v;
}
Point pointPolyTangent(const Polygon &pt, Point Q, int dir, int lo, int hi) {
    while (hi - lo > 1) {
        int mid = (lo + hi) / 2;
        bool pvs = isGood(pt[mid], pt[mid - 1], Q, dir);
        bool nxt = isGood(pt[mid], pt[mid + 1], Q, dir);

        if (pvs && nxt) return pt[mid];
        if (!(pvs || nxt)) {
            Point p1 = pointPolyTangent(pt, Q, dir, mid + 1, hi);
            Point p2 = pointPolyTangent(pt, Q, dir, lo, mid - 1);
            return better(p1, p2, Q, dir);
        }

        if (!pvs) {
            if (orient(Q, pt[mid], pt[lo]) == dir)
                hi = mid - 1;
            else if (better(pt[lo], pt[hi], Q, dir) == pt[lo])
                hi = mid - 1;
            else
                lo = mid + 1;
        }
        if (!nxt) {
            if (orient(Q, pt[mid], pt[lo]) == dir)
                lo = mid + 1;
            else if (better(pt[lo], pt[hi], Q, dir) == pt[lo])
                hi = mid - 1;
            else
                lo = mid + 1;
        }
    }

    Point ret = pt[lo];
    for (int i = lo + 1; i <= hi; i++) ret = better(ret, pt[i], Q, dir);
    return ret;
}
// [ACW, CW] Tangent
pair<Point, Point> pointPolyTangents(const Polygon &pt, Point Q) {
    int n = pt.size();
    Point acw_tan = pointPolyTangent(pt, Q, ACW, 0, n - 1);
    Point cw_tan = pointPolyTangent(pt, Q, CW, 0, n - 1);
    return make_pair(acw_tan, cw_tan);
}

\end{lstlisting}
\subsection{Three Dimensional}
\begin{lstlisting}

point get_perp(point p){ // returns a random perpendicular line to the vector p
    assert(sgn(norm(p)));

    point ret = point(-p.y, p.x, 0);
    if(sgn(norm(ret))) return ret;

    ret = point(0, -p.z, p.y);
    if(sgn(norm(ret))) return ret;

    assert(false)
}

struct plane{ // Caution: directed plane, directed on the direction of (p2 x p3)

	point n; // {a, b, c}
	double d; //ax + by + cz = d
	// d = n . p [ where p is any point on the plane ]

	plane(){;}
	plane(point _n, double _d){
        n = _n;
        d = _d;
	}
	plane(point p1, point p2, point p3){
		n = crsp(p2 - p1, p3 - p1);
		if(norm(n) < eps) {assert(false);} // doesn't define a plance
		d = dotp(p1, n);
	}

    //Preserves the direction
	point get_p1(){ return univ(n) * d / norm(n);}
	point get_p2(){ return get_p1() + get_perp(n);}
	point get_p3(){ return crsp(n, get_p2() - get_p1()) + get_p1();}

	int get_side(point p){ return sgn(dotp(n, p) - d);} ///OK
	double sgn_dist(point p) {return (dotp(n, p) - d) / norm(n);}
	double dist(point p) {return fabs(sgn_dist(p));}

	point project(point p){ return p - sgn_dist(p) * univ(n);}
	point reflect(point p){ return p - 2 * sgn_dist(p) * univ(n);}

    ///OK
    point get_coords(point p){ // "2-D"-fies the plane. All points on this plane have z = 0
        point O = get_p1();
        point ox = univ(get_p2() - O);
        point oy = univ(get_p3() - O);
        point oz = univ(n);
        p = p - O;
        return {dotp(p, ox), dotp(p, oy), dotp(p, oz)};
    }
};

plane translate(plane p, point t) {return {p.n, p.d + dotp(p.n, t)};}
plane shiftUp(plane p, double d) {return {p.n, p.d + d * norm(p.n)};}


point projection(point p, point st, point ed) { return dotp(ed - st, p - st) / norm(ed - st) * univ(ed - st) + st;} //OK
point extend(point st, point ed, double len) { return ed + univ(ed-st) * len;} //OK

point rtt(point axis, point p, double theta){
    axis = univ(axis);
    return p * cos(theta) +  sin(theta) * crsp(axis, p) + axis * (1-cos(theta)) * dotp(axis, p);
} //OK

point segmentProjection(point p, point st, point ed)
{
    double d = dotp(p - st, ed - st) / norm(ed - st);
    if(d < 0) return st;
    if(d > norm(ed - st) + eps) return ed;
    return st + univ(ed - st) * d;
} //OK

double distPointSegment(point p, point st, point ed) {return norm(p - segmentProjection(p, st, ed)); } //OK
double distPointLine( point P, point st, point ed) { return norm( projection(P, st, ed) - P ); } //OK

double pointPlanedist(plane P, point q){ return fabs(dotp(P.n, q) - P.d) / norm(P.n);}
double pointPlanedist(point p1, point p2, point p3, point q){ return pointPlanedist(plane(p1,p2,p3), q); } //OK

point reflection(point p, point st, point ed){
    point proj = projection(p, st, ed);
    if(p != proj) return extend(p, proj, norm(p - proj));
    return proj;
} //OK

bool coplanar(point p1, point p2, point p3, point q)
{
    p2 = p2-p1, p3 = p3-p1, q = q-p1;
    if( fabs( dotp(q, crsp(p2, p3)) ) < eps ) return true;
    return false;
}

int linePlaneIntersection(point u, point v, point l, point m, point r, point &x){
    /*
        -> l, m, r defines the plane
        -> u, v defines the line
        -> returns 0 when does not intersect
        -> returns 1 when there exists one unique common point
        -> returns -1 when there exists infinite number of common point
    */

    assert(l != m && m != r && l != r && u != v);
    if(coplanar(l, m, r, u) && coplanar(l, m, r, v)) return -1;
    l = l - m;
    r = r - m;
    u = u - m;
    v = v - m;

    point C = crsp(l, r);
    double denom = dotp(v - u, C);
    if(fabs(denom) < eps) return 0;

    double alpha = -dotp(C, u) / denom;
    x = u + (v - u) * alpha + m;

    return 1;
}

double angle(point u, point v) { return acos( max(-1.0, min(1.0, dotp(u, v) / (norm(u) * norm(v)))) );}

struct line3d{ //directed
    point d, o; // dir = direction, o = online point

    line3d(point p, point q){
        d = q - p;
        o = p;
        assert(sgn(norm(d)));
    }

    line3d(plane p1, plane p2){
        d = crsp(p1.n, p2.n);
        o = (crsp(p2.n*p1.d - p1.n*p2.d, d))/sq(d);
    }

    point get_p1(){return o;}
    point get_p2(){return o + d;}

    double dist(point p){ return norm(crsp(d, p - o)) / norm(d);};
    point project(point p){ return projection(p, o, o + d); }
    point reflect(point p) {return reflection(p, o, o + d);}
};

line3d perpThrough(plane p, point o){return line3d(o, o + p.n);}
plane perpThrough(line3d l, point o){return plane(l.d, dotp(l.d, o));}

double dist(line3d l1, line3d l2) {
    point n = crsp(l1.d, l2.d);
    if (!sgn(norm(n))) return l1.dist(l2.o);
    return abs(dotp(l2.o-l1.o, n))/norm(n);
}

point closestOnL1(line3d l1, line3d l2) {
    point n2 = crsp(l2.d, crsp(l1.d, l2.d));
    return l1.o + (l1.d * (dotp(l2.o-l1.o, n2))) / dotp(l1.d,n2);
}

double angle(plane p1, plane p2){return angle(p1.n, p2.n);}
bool isparallel(plane p1, plane p2){return !sgn(norm(crsp(p1.n, p2.n)));}
bool isperp(plane p1, plane p2) {return !sgn(dotp(p1.n, p2.n));}


double angle(line3d l1, line3d l2){return angle(l1.d, l2.d);}
bool isparallel(line3d l1, line3d l2){return !sgn(norm(crsp(l1.d, l2.d)));}
bool isperp(line3d l1, line3d l2) {return !sgn(dotp(l1.d, l2.d));}


double angle(plane p, line3d l) {return pi/2 - angle(p.n, l.d);}
bool isParallel(plane p, line3d l) {return !sgn(dotp(p.n, l.d));}
bool isPerpendicular(plane p, line3d l) {return !sgn(norm(crsp(p.n, l.d)));}

point vector_area2(vector <point> &poly){
    point S = {0, 0, 0};
    for(int i = 0; i < (int) poly.size(); i++)
        S = S + crsp(poly[i], poly[ (i + 1) % poly.size()]);
    return S;
}

double area(vector < point > &poly){ // All points must be co-planer
    return norm(vector_area2(poly)) * 0.5;
}

/*
    Polyhedrons
*/

bool operator <(point p, point q) { ///OK
    return tie(p.x, p.y, p.z) < tie(q.x, q.y, q.z);
}

struct edge {
    int v;
    bool same; // = is the common edge in the same order?
};

// Given a series of faces (lists of points), reverse some of them
// so that their orientations are consistent [ every face then will point in the same direction, inside / outside ]
void reorient(vector< vector<point> > &fs) {
    int n = fs.size();
    // Find the common edges and create the resulting graph
    vector< vector<edge>    > g(n);
    map<pair<point,point>, int> es;
    for (int u = 0; u < n; u++) {
        for (int i = 0, m = fs[u].size(); i < m; i++) {
            point a = fs[u][i], b = fs[u][(i+1)%m];
            // Let look at edge [AB]
            if (es.count({a,b})) { // seen in same order
                int v = es[{a,b}];
                g[u].push_back({v,true});
                g[v].push_back({u,true});
            }
            else if (es.count({b,a})) { // seen in different order
                    int v = es[{b,a}];
                    g[u].push_back({v,false});
                    g[v].push_back({u,false});
            }
            else es[{a,b}] = u;
        }
    }

    vector<bool> vis(n,false), flip(n);
    flip[0] = false;
    queue<int> q;
    q.push(0);
    while (!q.empty()) {
        int u = q.front();
        q.pop();
        for (edge e : g[u]) {
            if (!vis[e.v]) {
                vis[e.v] = true;
                // If the edge was in the same order,
                // exactly one of the two should be flipped
                flip[e.v] = (flip[u] ^ e.same);
                q.push(e.v);
            }
        }
    }

    for (int u = 0; u < n; u++)
        if (flip[u])
            reverse(fs[u].begin(), fs[u].end());
}

double volume(vector< vector<point> > fs) {
    double vol6 = 0.0;
    for (vector<point> f : fs)
        vol6 += dotp(vector_area2(f), f[0]);
    return abs(vol6) / 6.0;
}


/*
    Spherical Co-ordinate System
*/

point sph(double r, double lat, double lon) { // lat, lon in degrees
    lat *= pi/180, lon *= pi/180;
    return {r*cos(lat)*cos(lon), r*cos(lat)*sin(lon), r*sin(lat)};
}

double greatCircleDist(point o, double r, point a, point b) {
    return r * angle(a-o, b-o);
}

int main()
{

    plane p = {point(0, 0, 10), point(0, 1, 10), point(1, 0, 10)};
//    plane q = {p.get_p1(), p.get_p2(), p.get_p3()};
    double d = dotp(p.get_p2() - p.get_p1(), p.get_p3() - p.get_p1());

    D(eq(d, 0))
    return 0;
}

\end{lstlisting}
\section{Graph}
\subsection{2 Satisfiability}
\begin{lstlisting}
/*
  2-Sat Note: Assign true or false values to n variables in order to satisfy
  a system of constraints on pairs of variables.
  E.g: (x1 or !x2) and (x2 or x3) and (!x3 or !x3)

  x1 = true
  x2 = true
  x3 = false
  is a solution to make the above formula true.
  MAX must be equal to the maximum number of variables.
  n passed in init() is the number of variables.
  O(V+E)

  !a is represented as neg(a).
  example xor:
  |a|b|
  |0|0| x or(a,b)
  |0|1|
  |1|0|
  |1|1| x or(!a, !b)
  do OR of negation of values of variables for each undesired situation
  to make it impossible.
*/

struct two_sat {
  int n, id;
  vector<int> g[2 * MAX], rg[2 * MAX], order, st;
  bool state[2 * MAX], vis[2 * MAX];
  int scc[2 * MAX];

  void init(int _n) {
    n = _n;
    for (int i = 0; i <= 2 * n; i++) {
      g[i].clear(), rg[i].clear();
      state[i] = vis[i] = false;
      scc[i] = -1;
    }
    st.clear(), order.clear();
  }

  void add_edge(int u, int v) {
    g[u].pb(v);
    rg[v].pb(u);
  }

  void OR(int u, int v) {
    add_edge(neg(u), v);
    add_edge(neg(v), u);
  }

  void XOR(int u, int v) {
    OR(u, v);
    OR(neg(u), neg(v));
  }

  void ForceTrue(int u) {
    add_edge(neg(u), u);
  }

  void ForceFalse(int u) {
    add_edge(u, neg(u));
  }

  void imply(int u, int v) {
    OR(neg(u), v);
  }

  int neg(int u) {
    if (u <= n) return u + n;
    return u - n;
  }

  void dfs(int u, vii g[], bool topsort) {
    vis[u] = true;
    for (int v : g[u]) {
      if (!vis[v]) dfs(v, g, topsort);
    }
    if (topsort) st.pb(u);
    else scc[u] = id, order.pb(u);
  }

  void build_scc() {
    for (int i = 1; i <= 2 * n; i++) {
      if (!vis[i]) dfs(i, g, true);
    }
    reverse(st.begin(), st.end());
    fill(vis, vis + 2 * n + 1, false);
    for (int u : st) {
      if (!vis[u]) id++, dfs(u, rg, false);
    }
  }

  bool solve() {
    build_scc();
    for (int i = 1; i <= n; i++) {
      if (scc[i] == scc[i + n]) return false;
    }
    for (int i = (int)order.size() - 1; i >= 0; i--) {
      int u = order[i];
      if (state[neg(u)] == false) state[u] = true;
    }
    return true;
  }
} solver;
\end{lstlisting}
\subsection{Block Cut Tree}
\begin{lstlisting}
bool ap[MAX];
int id[MAX], koyta[MAX];
int d[MAX], low[MAX];
bool vis[MAX];
vii g[MAX], tree[MAX];
int d_t;
stack<int>st;
vector<vector<int>>comp;
void articulation(int u, int p) {
  vis[u] = true;
  d[u] = low[u] = ++d_t;
  int child = 0; st.push(u);
  for (int v : g[u]) {
    if (v == p) continue;
    if (!vis[v]) {
      child++;
      articulation(v, u);
      low[u] = min(low[u], low[v]);
      if (p == -1 && child > 1) ap[u] = true;
      if (low[v] >= d[u]) {
        if (p != -1) ap[u] = true;
        comp.pb({u}); int top;
        do {
          top = st.top(); st.pop();
          comp.back().pb(top);
        } while (top != v);
      }
    } else low[u] = min(low[u], d[v]);
  }
}
int node = 0;
void make_tree(int n) {
  for (int i = 1; i <= n; i++) {
    if (ap[i]) id[i] = ++node;
  }
  for (int i = 0; i < comp.size(); i++) {
    ++node;
    int cnt = 0;
    for (int u : comp[i]) {
      if (ap[u]) tree[node].pb(id[u]), tree[id[u]].pb(node), koyta[id[u]] = 1;
      else id[u] = node, cnt++;
    }
    koyta[node] = cnt;
  }
}
\end{lstlisting}
\subsection{Bridge Tree}
\begin{lstlisting}
vector<int> tree[MAX];
bool vis[MAX];
int d[MAX], low[MAX];
int id[MAX];
int d_t;

struct edge {
  int v, rev;
  edge() {}
  edge(int v, int rev) : v(v), rev(rev) {}
};
vector<edge>g[MAX];
vector<bool>is_bridge[MAX];
queue<int>q[MAX];
int comp = 1;

void add_edge(int u, int v) {
  edge _u = edge(v, g[v].size());
  edge _v = edge(u, g[u].size());
  g[u].pb(_u);
  g[v].pb(_v);
  is_bridge[u].pb(false);
  is_bridge[v].pb(false);
}

void bridge(int u, int p) {
  vis[u] = true;
  d[u] = low[u] = ++d_t;

  for (int i = 0; i < g[u].size(); i++) {
    edge e = g[u][i]; int v = e.v;
    if (v == p) continue;
    if (!vis[v]) {
      bridge(v, u);
      low[u] = min(low[v], low[u]);
      if (low[v] > d[u]) {
        is_bridge[u][i] = true;
        is_bridge[v][e.rev] = true;
      }
    } else low[u] = min(low[u], d[v]);
  }
}

void make_tree(int node) {
  int cur = comp; q[cur].push(node);
  vis[node] = true; id[node] = cur;
  while (!q[cur].empty()) {
    int u = q[cur].front(); q[cur].pop();
    for (int i = 0; i < g[u].size(); i++) {
      edge e = g[u][i]; int v = e.v;
      if (vis[v]) continue;
      if (is_bridge[u][i]) {
        comp++;
        tree[cur].pb(comp);
        tree[comp].pb(cur);
        make_tree(v);
      } else {
        q[cur].push(v);
        vis[v] = true; id[v] = cur;
      }
    }
  }
}
\end{lstlisting}
\subsection{Centroid Decomposition}
\begin{lstlisting}
// problem: calculate the sum of number of distinct colors in the path between any two pair of nodes
//centroid decomposition (res[i] contains the sum of numbers of distinct colors in all paths from i)
set<int>g[MAX];
int col[MAX], child[MAX], used[18][MAX];
ll ans[MAX], res[MAX];
int sz = 0, uniq = 0, n;
bool vis[MAX];
void dfs(int u, int p) {
  sz++; child[u] = 1;
  for (auto v : g[u]) {
    if (v != p) {
      dfs(v, u);
      child[u] += child[v];
    }
  }
}
int get_centroid(int u, int p) {
  for (auto v : g[u]) {
    if (v != p && child[v] > sz / 2) return get_centroid(v, u);
  }
  return u;
}
void add(int u, int p, int depth, int centroid) {
  bool check = false; child[u] = 1;
  if (!vis[col[u]]) {
    uniq++; check = true;
    vis[col[u]] = true;
  }
  ans[centroid] += uniq;
  for (auto v : g[u]) {
    if (v != p) {
      add(v, u, depth, centroid);
      child[u] += child[v];
    }
  }
  if (check) {
    uniq--;
    used[depth][col[u]] += child[u];
    vis[col[u]] = false;
  }
}
void del(int u, int p, int depth, int centroid) {
  bool check = false;
  if (!vis[col[u]]) {
    uniq++;
    used[depth][col[u]] -= child[u];
    vis[col[u]] = true; check = true;
  }
  ans[centroid] -= uniq;
  for (auto v : g[u]) {
    if (v != p) del(v, u, depth, centroid);
  }
  child[u] = 0;
  if (check) uniq--; vis[col[u]] = false;
}
void solve(int u, int p, int depth, int centroid) {
  ans[u] += (ans[p] + child[centroid] - used[depth][col[u]]);
  res[u] += ans[u];
  int temp = used[depth][col[u]];
  used[depth][col[u]] = child[centroid];
  for (auto v : g[u]) {
    if (v != p) solve(v, u, depth, centroid);
  }
  ans[u] = 0;
  used[depth][col[u]] = temp;
}
void reset_col(int u, int p, int depth) {
  used[depth][col[u]] = 0;
  for (auto v : g[u]) {
    if (v != p) reset_col(v, u, depth);
  }
}
void decompose(int u, int depth) {
  sz = 0;
  uniq = 0;
  dfs(u, -1);
  int centroid = get_centroid(u, -1);
  reset_col(centroid, -1, depth);
  add(centroid, -1, depth, centroid); ///get ans for centroid and get the number of paths where each color is used
  res[centroid] += ans[centroid];
  uniq++;
  vis[col[centroid]] = true;
  for (auto v : g[centroid]) {
    child[centroid] -= child[v];
    ///remove all contribution of the subtree of v
    del(v, centroid, depth, centroid);
    used[depth][col[centroid]] = child[centroid];
    solve(v, centroid, depth, centroid);
    ///add back the contribution of the subtree of v
    add(v, centroid, depth, centroid);
    child[centroid] += child[v];
  }
  uniq--;
  vis[col[centroid]] = false;
  for (auto it = g[centroid].begin(); it != g[centroid].end(); it++) {
    g[*it].erase(centroid);
    decompose(*it, depth + 1);
  }
}
int arr[MAX];
int main() {
  fastio;
  cin >> n;
  for (int i = 1; i <= n; i++) cin >> col[i];
  for (int i = 0; i < n - 1; i++) {
    int u, v;
    cin >> u >> v;
    g[u].insert(v); g[v].insert(u);
  }
  decompose(1, 0);
  for (int i = 1; i <= n; i++) cout << res[i] << "\n";
}
\end{lstlisting}
\subsection{Dinic Max-Flow}
\begin{lstlisting}
int src, sink;
int dis[MAX], q[MAX], work[MAX];
int n, m, nodes;

struct edge {
  int v, rev, cap, flow;
  edge() {}
  edge(int v, int rev, int cap) : v(v), rev(rev), cap(cap), flow(0) {}
};
vector<edge>g[MAX];

void add_edge(int u, int v, int cap, int rev = 0) {
  edge _u = edge(v, g[v].size(), cap);
  edge _v = edge(u, g[u].size(), rev);
  g[u].pb(_u);
  g[v].pb(_v);
}

bool dinic_bfs(int s) {
  fill(dis, dis + nodes, -1);
  dis[s] = 0;
  int id = 0;
  q[id++] = s;

  for (int i = 0; i < id; i++) {
    int u = q[i];
    for (int j = 0; j < g[u].size(); j++) {
      edge &e = g[u][j];
      if (dis[e.v] == -1 && e.flow < e.cap) {
        dis[e.v] = dis[u] + 1;
        q[id++] = e.v;
      }
    }
  }
  return dis[sink] >= 0;
}

int dinic_dfs(int u, int f) {
  if (u == sink) return f;
  for (int &i = work[u]; i < g[u].size(); i++) {
    edge &e = g[u][i];
    if (e.cap <= e.flow) continue;
    if (dis[e.v] == dis[u] + 1) {
      int flow = dinic_dfs(e.v, min(f, e.cap - e.flow));
      if (flow) {
        e.flow += flow;
        g[e.v][e.rev].flow -= flow;
        return flow;
      }
    }
  }
  return 0;
}

int max_flow(int _src, int _sink) {
  src = _src;
  sink = _sink;
  int ret = 0;
  while (dinic_bfs(src)) {
    fill(work, work + nodes, 0);
    while (int flow = dinic_dfs(src, INT_MAX)) {
      ret += flow;
    }
  }
  return ret;
}
\end{lstlisting}
\subsection{Flow with Lower Bound}
\begin{lstlisting}
///flow with demand(lower bound) only for DAG
//create new src and sink
//add_edge(new src, u, sum(in_demand[u]))
//add_edge(u, new sink, sum(out_demand[u]))
//add_edge(old sink, old src, inf)
// if (sum of lower bound == flow) then demand satisfied
//flow in every edge i = demand[i] + e.flow
\end{lstlisting}
\subsection{Heavy Light Decomposition}
\begin{lstlisting}
int arr[MAX], n;
vector<int> parent, depth, heavy, head, pos;
int cur_pos, sub[MAX];
int tree[4 * MAX];
vii g[MAX];
void update(int now, int L, int R, int idx, int val) {
  if (L == R) {
    tree[now] = val;
    return;
  }
  int mid = (L + R) / 2;
  if (idx <= mid) update(now << 1, L, mid, idx, val);
  else update( (now << 1) | 1, mid + 1, R, idx, val);
  tree[now] = tree[now << 1] + tree[(now << 1) | 1];
}
ll segtree_query(int now, int L, int R, int i, int j) {
  if (R < i || L > j) return 0;
  if (L >= i && R <= j) return tree[now];
  int mid = (L + R) / 2;
  return segtree_query(now << 1, L, mid, i, j) + segtree_query((now << 1) | 1, mid + 1, R, i, j);
}
int dfs(int u) {
  sub[u] = 1;
  int mx_size = 0;
  for (int v : g[u]) {
    if (v != parent[u]) {
      parent[v] = u, depth[v] = depth[u] + 1;
      int v_size = dfs(v);
      sub[u] += v_size;
      if (v_size > mx_size) {
        mx_size = v_size;
        heavy[u] = v;
      }
    }
  }
  return sub[u];
}
void decompose(int u, int h) {
  head[u] = h, pos[u] = cur_pos++;
  if (heavy[u] != -1) decompose(heavy[u], h);
  for (int v : g[u]) {
    if (v != parent[u] && v != heavy[u]) decompose(v, v);
  }
}
void init(int n) {
  parent = vector<int>(n, -1);
  depth = vector<int>(n);
  heavy = vector<int>(n, -1);
  head = vector<int>(n);
  pos = vector<int>(n);
  cur_pos = 1;
  dfs(1); decompose(1, 1);
}
ll query(int a, int b) {
  ll res = 0;
  for (; head[a] != head[b]; b = parent[head[b]]) {
    if (depth[head[a]] > depth[head[b]]) swap(a, b);
    ll cur_heavy_path_res = segtree_query(1, 1, n, pos[head[b]], pos[b]);
    res += cur_heavy_path_res;
  }
  if (depth[a] > depth[b]) swap(a, b);
  ll last_heavy_path_res = segtree_query(1, 1, n, pos[a], pos[b]);
  res += last_heavy_path_res;
  return res;
}
\end{lstlisting}
\subsection{K-th Root of a Permutation}
\begin{lstlisting}
vector<vector<int>> decompose(vector<int> &p) {
  int n = p.size();
  vector<vector<int>> cycles;
  vector<bool> vis(n, 0);
  for (int i = 0; i < n; i++) {
    if (!vis[i]) {
      vector<int> v;
      while (!vis[i]) {
        v.push_back(i);
        vis[i] = 1;
        i = p[i];
      }
      cycles.push_back(v);
    }
  }
  return cycles;
}
vector<int> restore(int n, vector<vector<int>> &cycles) {
  vector<int> p(n);
  for (auto v : cycles) {
    int m = v.size();
    for (int i = 0; i < m; i++) p[v[i]] = v[(i + 1) % m];
  }
  return p;
}
//cycle decomposition of the k-th power of p
vector<vector<int>> power(vector<int> &p, int k) {
  int n = p.size();
  auto cycles = decompose(p);
  vector<vector<int>> ans;
  for (auto v : cycles) {
    int len = v.size(), g = __gcd(k, len); //g cycles of len / g length
    for (int i = 0; i < g; i++) {
      vector<int> w;
      for (int j = i, cnt = 0; cnt < len / g; cnt++, j = (j + k) % len) {
        w.push_back(v[j]);
      }
      ans.push_back(w);
    }
  }
  return ans;
}
//cycle decomposition of the k-th root of p with minimum disjoint cycles
//if toggle = 1, then the parity of number of cycles will be different from the other(toggle = 0) if possible
//returns empty vector if no solution exists
vector<vector<int>> root(vector<int> &p, int k, int toggle = 0) {
  int n = p.size();
  vector<vector<int>> cycles[n + 1];
  auto d = decompose(p);
  for (auto v : d) cycles[(int)v.size()].push_back(v);
  vector<vector<int>> ans;
  for (int len = 1; len <= n; len++) {
    if (cycles[len].empty()) continue;
    int tmp = k, t = 1, x = __gcd(len, tmp);
    while (x != 1) {
      t *= x;
      tmp /= x;
      x = __gcd(len, tmp);
    }
    if ((int)cycles[len].size() % t != 0) {
      ans.clear();
      return ans; //no solution exists
    }
    int id = 0;
    //we can merge t * z cycles iff tmp % z === 0
    if (toggle && tmp % 2 == 0 && (int)cycles[len].size() >= 2 * t) {
      int m = 2 * t * len;
      vector<int> cycle(m);
      for (int x = 0; x < 2 * t; x++) { //merging 2t cycles to perform the toggle
        for (int y = 0; y < len; y++) {
          cycle[(x + 1LL * y * k) % m] = cycles[len][x][y];
        }
      }
      ans.push_back(cycle);
      id = 2 * t;
      toggle = 0;
    }
    for (int i = id; i < (int)cycles[len].size(); i += t) {
      int m = t * len;
      vector<int> cycle(m);
      for (int x = 0; x < t; x++) { //merging t cycles
        for (int y = 0; y < len; y++) {
          cycle[(x + 1LL * y * k) % m] = cycles[len][i + x][y];
        }
      }
      ans.push_back(cycle);
    }
  }
  return ans;
}
//minimum swaps to obtain this perm from unit perm
vector<pair<int, int>> transpositions(vector<vector<int>> &cycles) {
  vector<pair<int, int>> ans;
  for (auto v : cycles) {
    int m = v.size();
    for (int i = m - 1; i > 0; i--) ans.push_back({v[0], v[i]});
  }
  return ans;
}
int32_t main() {
  ios_base::sync_with_stdio(0);
  cin.tie(0);
  int n, l, k;
  cin >> n >> l >> k;
  vector<int> p(n);
  for (auto &x : p) cin >> x, --x;
  auto a = root(p, k);
  if (a.empty()) return cout << "no solution\n", 0;
  auto t = transpositions(a);
  if (t.size() % 2 != l % 2) {
    a = root(p, k, 1);
    t = transpositions(a);
  }
  if (t.size() % 2 != l % 2 || t.size() > l) return cout << "no solution\n", 0;
  auto z = restore(n, a);
  auto w = power(z, k);
  auto x = restore(n, w);
  assert(p == x);
  for (auto x : t) cout << x.first + 1 << ' ' << x.second + 1 << '\n';
  for (int i = t.size(); i < l; i++) cout << 1 << ' ' << 2 << '\n';
  return 0;
}

\end{lstlisting}
\subsection{Lowest Common Ancestor}
\begin{lstlisting}
int lvl[MAX], P[MAX][25];
void dfs(int u, int par, int d) {
  lvl[u] = d;
  P[u][0] = par;
  for (int v : g[u]) {
    if (v == par) continue;
    dfs(v, u, d + 1);
  }
}

void init() {
  for (int j = 0; j < 25; j++) {
    for (int i = 0; i <= n; i++) P[i][j] = -1;
  }

  dfs(1, -1, 0);

  for (j = 1; j < 25; j++) {
    for (int i = 1; i <= n; i++) {
      if (P[i][j - 1] != -1) {
        P[i][j] = P[ P[i][j - 1] ][j - 1];
        ///to find max weight between two edges
        // weight[i][j] = max(weight[i][j-1], weight[p[i][j-1]][j-1]);
      }
    }
  }
}


int lca(int u, int v) {
  int i, lg;
  if (lvl[u] < lvl[v]) swap(u, v);

  for (lg = 0; (1 << lg) <= lvl[u]; lg++);
  lg--;

  for (i = lg; i >= 0; i--) {
    if (lvl[u] - (1 << i) >= lvl[v]) {
      u = P[u][i];
    }
  }

  if (u == v) return u;

  for (i = lg; i >= 0; i--) {
    if (P[u][i] != -1 && P[u][i] != P[v][i]) {
      u = P[u][i], v = P[v][i];
      // ret = max(ret, weight[u][i]);
      // ret = max(ret, weight[v][i]);
    }
  }
  // ret = max(ret, weight[u][0]);
  return P[u][0];
}

//Get the ancestor of node "u"
//which is "dis" distance above.
int getAncestor(int u, int dis) {
  dis = lvl[u] - dis;
  int i, lg = 0;
  for (; (1 << lg) <= lvl[u]; lg++) continue;
  lg--;

  for (i = lg; i >= 0; i--) {
    if (lvl[u] - (1 << i) >= dis) {
      u = P[u][i];
    }
  }
  return u;
}

//returns the distance between
//two nodes "u" and "v".
int dis(int u, int v) {
  if (lvl[u] < lvl[v]) swap(u, v);
  int p = lca(u, v);
  return lvl[u] + lvl[v] - 2 * lvl[p];
}
\end{lstlisting}
\subsection{Min Cost Max Flow}
\begin{lstlisting}
mt19937 rnd(chrono::steady_clock::now().time_since_epoch().count());
struct edge {
  int v, rev;
  ll cap, cost, flow;
  edge() {}
  edge(int v, int rev, ll cap, ll cost) : v(v), rev(rev), cap(cap), cost(cost), flow(0) {}
};
struct mcmf {
  int src, sink, nodes;
  vector<int> par, idx, Q;
  vector<bool> inq;
  vector<ll> dis;
  vector<vector<edge>> g;
  mcmf() {}
  mcmf(int src, int sink, int nodes) : src(src), sink(sink), nodes(nodes),
    par(nodes), idx(nodes), inq(nodes),
    dis(nodes), g(nodes), Q(10000005) {} // use Q(nodes) if not using random
  void add_edge(int u, int v, ll cap, ll cost, bool directed = true) {
    edge _u = edge(v, g[v].size(), cap, cost);
    edge _v = edge(u, g[u].size(), 0, -cost);
    g[u].pb(_u);
    g[v].pb(_v);
    if (!directed) add_edge(v, u, cap, cost, true);
  }
  bool spfa() {
    for (int i = 0; i < nodes; i++) {
      dis[i] = inf, inq[i] = false;
    }
    int f = 0, l = 0;
    dis[src] = 0, par[src] = -1, Q[l++] = src, inq[src] = true;
    while (f < l) {
      int u = Q[f++];
      for (int i = 0; i < g[u].size(); i++) {
        edge &e = g[u][i];
        if (e.cap <= e.flow) continue;
        if (dis[e.v] > dis[u] + e.cost) {
          dis[e.v] = dis[u] + e.cost;
          par[e.v] = u, idx[e.v] = i;
          if(!inq[e.v]) inq[e.v] = true, Q[l++] = e.v;
          // if (!inq[e.v]) {
          //   inq[e.v] = true;
          //   if (f && rnd() & 7) Q[--f] = e.v;
          //   else Q[l++] = e.v;
          // }
        }
      }
      inq[u] = false;
    }
    return (dis[sink] != inf);
  }
  pair<ll, ll> solve() {
    ll mincost = 0, maxflow = 0;
    while (spfa()) {
      ll bottleneck = inf;
      for (int u = par[sink], v = idx[sink]; u != -1; v = idx[u], u = par[u]) {
        edge &e = g[u][v];
        bottleneck = min(bottleneck, e.cap - e.flow);
      }
      for (int u = par[sink], v = idx[sink]; u != -1; v = idx[u], u = par[u]) {
        edge &e = g[u][v];
        e.flow += bottleneck;
        g[e.v][e.rev].flow -= bottleneck;
      }
      mincost += bottleneck * dis[sink], maxflow += bottleneck;
    }
    return make_pair(mincost, maxflow);
  }
}; 
// want to minimize cost and don't care about flow
// add edge from sink to dummy sink (cap = inf, cost = 0)
// add edge from source to sink (cap = inf, cost = 0)
// run mcmf, cost returned is the minimum cost
\end{lstlisting}
\subsection{Tree Isomorphism}
\begin{lstlisting}
mp["01"] = 1;
ind = 1;
int dfs(int u, int p) {
  int cnt = 0;
  vector<int>vs;
  for (auto v : g1[u]) {
    if (v != p) {
      int got = dfs(v, u);
      vs.pb(got);
      cnt++;
    }
  }
  if (!cnt) return 1;

  sort(vs.begin(), vs.end());
  string s = "0";
  for (auto i : vs) s += to_string(i);
  vs.clear();
  s.pb('1');
  if (mp.find(s) == mp.end()) mp[s] = ++ind;
  int ret = mp[s];
  return ret;
}
\end{lstlisting}
\section{Math}
\subsection{Berlekamp Massey}
\begin{lstlisting}
struct berlekamp_massey { // for linear recursion
  typedef long long LL;
  static const int SZ = 2e5 + 5;
  static const int MOD = 1e9 + 7; /// mod must be a prime
  LL m , a[SZ] , h[SZ] , t_[SZ] , s[SZ] , t[SZ];
  // bigmod goes here
  inline vector <LL> BM( vector <LL> &x ) {
    LL lf , ld;
    vector <LL> ls , cur;
    for ( int i = 0; i < int(x.size()); ++i ) {
      LL t = 0;
      for ( int j = 0; j < int(cur.size()); ++j ) t = (t + x[i - j - 1] * cur[j]) % MOD;
      if ( (t - x[i]) % MOD == 0 ) continue;
      if ( !cur.size() ) {
        cur.resize( i + 1 );
        lf = i; ld = (t - x[i]) % MOD;
        continue;
      }
      LL k = -(x[i] - t) * bigmod( ld , MOD - 2 , MOD ) % MOD;
      vector <LL> c(i - lf - 1);
      c.push_back( k );
      for ( int j = 0; j < int(ls.size()); ++j ) c.push_back(-ls[j] * k % MOD);
      if ( c.size() < cur.size() ) c.resize( cur.size() );
      for ( int j = 0; j < int(cur.size()); ++j ) c[j] = (c[j] + cur[j]) % MOD;
      if (i - lf + (int)ls.size() >= (int)cur.size() ) ls = cur, lf = i, ld = (t - x[i]) % MOD;
      cur = c;
    }
    for ( int i = 0; i < int(cur.size()); ++i ) cur[i] = (cur[i] % MOD + MOD) % MOD;
    return cur;
  }
  inline void mull( LL *p , LL *q ) {
    for ( int i = 0; i < m + m; ++i ) t_[i] = 0;
    for ( int i = 0; i < m; ++i ) if ( p[i] )
        for ( int j = 0; j < m; ++j ) t_[i + j] = (t_[i + j] + p[i] * q[j]) % MOD;
    for ( int i = m + m - 1; i >= m; --i ) if ( t_[i] )
        for ( int j = m - 1; ~j; --j ) t_[i - j - 1] = (t_[i - j - 1] + t_[i] * h[j]) % MOD;
    for ( int i = 0; i < m; ++i ) p[i] = t_[i];
  }
  inline LL calc( LL K ) {
    for ( int i = m; ~i; --i ) s[i] = t[i] = 0;
    s[0] = 1; if ( m != 1 ) t[1] = 1; else t[0] = h[0];
    while ( K ) {
      if ( K & 1 ) mull( s , t );
      mull( t , t ); K >>= 1;
    }
    LL su = 0;
    for ( int i = 0; i < m; ++i ) su = (su + s[i] * a[i]) % MOD;
    return (su % MOD + MOD) % MOD;
  }
  /// already calculated upto k , now calculate upto n.
  inline vector <LL> process( vector <LL> &x , int n , int k ) {
    auto re = BM( x );
    x.resize( n + 1 );
    for ( int i = k + 1; i <= n; i++ ) {
      for ( int j = 0; j < re.size(); j++ ) {
        x[i] += 1LL * x[i - j - 1] % MOD * re[j] % MOD; x[i] %= MOD;
      }
    }
    return x;
  }
  inline LL work( vector <LL> &x , LL n ) {
    if ( n < int(x.size()) ) return x[n] % MOD;
    vector <LL> v = BM( x ); m = v.size(); if ( !m ) return 0;
    for ( int i = 0; i < m; ++i ) h[i] = v[i], a[i] = x[i];
    return calc( n ) % MOD;
  }
} rec;
vector <LL> v;
void solve() {
  int n;
  cin >> n;
  cout << rec.work(v, n - 1) << endl;
}

\end{lstlisting}
\subsection{Chinese Remainder Theorem}
\begin{lstlisting}
// given a, b will find solutions for
// ax + by = 1
tuple <LL,LL,LL> EGCD(LL a, LL b){
    if(b == 0) return {1, 0, a};
    else{
        auto [x,y,g] = EGCD(b, a%b);
        return {y, x - a/b*y,g};
    }
}
// given modulo equations, will apply CRT
PLL CRT(vector <PLL> &v){
    LL V = 0, M = 1;
    for(auto &[v, m]:v){
        auto [x, y, g] = EGCD(M, m);
        if((v - V) % g != 0)
            return {-1, 0};
        V += x * (v - V) / g % (m / g) * M, M *= m / g;
        V = (V % M + M) % M;
    }
    return make_pair(V, M);
}\end{lstlisting}
\subsection{Derangements}
\begin{lstlisting}
array <int, N + 1> Drng;
void init(){
    Drng[0] = 1, Drng[1] = 0;
    for(int i = 2; i <= N; i++)
        Drng[i] = (LL) (i - 1) * (Drng[i - 1] + Drng[i - 2]) % mod;
}
int D(int n) { 
    return n < 0 ? 0 : Drng[n];
}\end{lstlisting}
\subsection{FFT in Mod}
\begin{lstlisting}
const int M = 1e9 + 7, B = sqrt(M) + 1;
vector<LL> anyMod(const vector<LL> &a, const vector<LL> &b) {
    int n = 1;
    while (n < a.size() + b.size())  n <<= 1;
    vector<CD> al(n), ar(n), bl(n), br(n);

    for (int i = 0; i < a.size(); i++)  al[i] = a[i] % M / B, ar[i] = a[i] % M % B;
    for (int i = 0; i < b.size(); i++)  bl[i] = b[i] % M / B, br[i] = b[i] % M % B;

    pairfft(al, ar); pairfft(bl, br);
//        fft(al); fft(ar); fft(bl); fft(br);

    for (int i = 0; i < n; i++) {
        CD ll = (al[i] * bl[i]), lr = (al[i] * br[i]);
        CD rl = (ar[i] * bl[i]), rr = (ar[i] * br[i]);
        al[i] = ll; ar[i] = lr;
        bl[i] = rl; br[i] = rr;
    }

    pairfft(al, ar, true); pairfft(bl, br, true);
//        fft(al, true); fft(ar, true); fft(bl, true); fft(br, true);

    vector<LL> ans(n);
    for (int i = 0; i < n; i++) {
        LL right = round(br[i].real()), left = round(al[i].real());;
        LL mid = round(round(bl[i].real()) + round(ar[i].real()));
        ans[i] = ((left % M) * B * B + (mid % M) * B + right) % M;
    }
    return ans;
}
\end{lstlisting}
\subsection{Fast Fourier Transformation}
\begin{lstlisting}
/**
1. Whenever possible remove leading zeros.
2. Custom Complex class may slightly improve performance.
3. Use pairfft to do two ffts of real vectors at once, slightly less accurate
than doing two ffts, but faster by about 30%.
4. FFT accuracy depends on answer. x <= 5e14 (double), x <= 1e18(long double)
   where x = max(ans[i]) for FFT, and x = N*mod for anymod **/
const double PI = acos(-1.0L);
int N;
vector<int> perm;
vector<CD> wp[2];

void precalculate(int n) {
    assert((n & (n - 1)) == 0);
    N = n;
    perm = vector<int> (N, 0);
    for (int k = 1; k < N; k <<= 1)
        for (int i = 0; i < k; i++)
            perm[i] <<= 1, perm[i + k] = 1 + perm[i];

    wp[0] = wp[1] = vector<CD>(N);
    for (int i = 0; i < N; i++) {
        wp[0][i] = CD(cos(2 * PI * i / N),  sin(2 * PI * i / N));
        wp[1][i] = CD(cos(2 * PI * i / N), -sin(2 * PI * i / N));
    }
}

void fft(vector<CD> &v, bool invert = false) {
    if (v.size() != perm.size())    precalculate(v.size());

    for (int i = 0; i < N; i++)
        if (i < perm[i])
            swap(v[i], v[perm[i]]);

    for (int len = 2; len <= N; len *= 2) {
        for (int i = 0, d = N / len; i < N; i += len) {
            for (int j = 0, idx = 0; j < len / 2; j++, idx += d) {
                CD x = v[i + j];
                CD y = wp[invert][idx] * v[i + j + len / 2];
                v[i + j] = x + y;
                v[i + j + len / 2] = x - y;
            }
        }
    }

    if (invert) {
        for (int i = 0; i < N; i++) v[i] /= N;
    }
}

void pairfft(vector<CD> &a, vector<CD> &b, bool invert = false) {
    int N = a.size();
    vector<CD> p(N);
    for (int i = 0; i < N; i++) p[i] = a[i] + b[i] * CD(0, 1);
    fft(p, invert);
    p.push_back(p[0]);

    for (int i = 0; i < N; i++) {
        if (invert) {
            a[i] = CD(p[i].real(), 0);
            b[i] = CD(p[i].imag(), 0);
        } else {
            a[i] = (p[i] + conj(p[N - i])) * CD(0.5, 0);
            b[i] = (p[i] - conj(p[N - i])) * CD(0, -0.5);
        }
    }
}

vector<LL> multiply(const vector<LL> &a, const vector<LL> &b) {
    int n = 1;
    while (n < a.size() + b.size())  n <<= 1;

    vector<CD> fa(a.begin(), a.end()), fb(b.begin(), b.end());
    fa.resize(n); fb.resize(n);

//        fft(fa); fft(fb);
    pairfft(fa, fb);
    for (int i = 0; i < n; i++) fa[i] = fa[i] * fb[i];
    fft(fa, true);

    vector<LL> ans(n);
    for (int i = 0; i < n; i++)     ans[i] = round(fa[i].real());
    return ans;
}\end{lstlisting}
\subsection{Fast Walsh Hadamord Transformation}
\begin{lstlisting}
#define bitwiseXOR 1
//#define bitwiseAND 2
//#define bitwiseOR 3

void FWHT(vector< LL >&p, bool inverse){
    LL n = p.size();
    assert((n&(n-1))==0);

    for (LL len = 1; 2*len <= n; len <<= 1) {
        for (LL i = 0; i < n; i += len+len) {
            for (LL j = 0; j < len; j++) {
                LL u = p[i+j];
                LL v = p[i+len+j];

                #ifdef bitwiseXOR
                p[i+j] = u+v;
                p[i+len+j] = u-v;
                #endif // bitwiseXOR

                #ifdef bitwiseAND
                if (!inverse) {
                    p[i+j] = v % MOD;
                    p[i+len+j] = (u+v) % MOD;
                } else {
                    p[i+j] = (-u+v) % MOD;
                    p[i+len+j] = u % MOD;
                }
                #endif // bitwiseAND

                #ifdef bitwiseOR
                if (!inverse) {
                    p[i+j] = u+v;
                    p[i+len+j] = u;
                } else {
                    p[i+j] = v;
                    p[i+len+j] = u-v;
                }
                #endif // bitwiseOR
            }
        }
    }

    #ifdef bitwiseXOR
    if (inverse) {
        //LL val=BigMod(n,MOD-2); //Option 2: Exclude
        for (LL i = 0; i < n; i++) {
            //assert(p[i]%n==0); //Option 2: Include
            //p[i] = (p[i]*val)%MOD; //Option 2: p[i]/=n;
            p[i]/=n;
        }
    }
    #endif // bitwiseXOR
}

vector<pair<int,int> >V[100005];
int dis[100005];

void dfs(int s,int pr){
    for(auto p:V[s]){
        if(p.first==pr) continue;
        dis[p.first]=dis[s]^p.second;
        dfs(p.first,s);
    }
}

int main(){
    int t;
    cin >> t;
    const int len=(1<<16);
    for(int tc=1;tc<=t;tc++){
        LL n;
        cin >> n;
        for(int i=1;i<=n-1;i++){
            int u,v,w;
            cin >> u >> v >> w;
            V[u].push_back({v,w});
            V[v].push_back({u,w});
        }
        dfs(1,0);
        vector<LL>a(len,0);
        for(int i=1;i<=n;i++) a[dis[i]]++;
        FWHT(a,false);
        for(int i=0;i<len;i++) a[i]*=a[i];
        FWHT(a,true);
        a[0]-=n;
        cout << "Case " << tc << ":\n";
        for(int i=0;i<len;i++) cout << a[i]/2 << '\n';
        memset(dis,0,sizeof(dis));
        for(int i=1;i<=n;i++) V[i].clear();
    }
}
\end{lstlisting}
\subsection{Gaussian Elimination}
\begin{lstlisting}
const double EPS = 1e-9;
const int INF = 2; // it doesn't actually have to be infinity or a big number

template <typename DT> int gauss (vector < vector<DT> > a, vector<DT> & ans) {
    int n = (int) a.size();
    int m = (int) a[0].size() - 1;

    vector<int> where (m, -1);
    for (int col=0, row=0; col<m && row<n; ++col) {
        int sel = row;
        for (int i=row; i<n; ++i)
            if (abs (a[i][col]) > abs (a[sel][col]))
                sel = i;
        if (abs (a[sel][col]) < EPS)
            continue;
        for (int i=col; i<=m; ++i)
            swap (a[sel][i], a[row][i]);
        where[col] = row;

        for (int i=0; i<n; ++i)
            if (i != row) {
                DT c = a[i][col] / a[row][col];
                for (int j=col; j<=m; ++j)
                    a[i][j] -= a[row][j] * c;
            }
        ++row;
    }

    ans.assign (m, 0);
    for (int i=0; i<m; ++i)
        if (where[i] != -1)
            ans[i] = a[where[i]][m] / a[where[i]][i];
    for (int i=0; i<n; ++i) {
        DT sum = 0;
        for (int j=0; j<m; ++j)
            sum += ans[j] * a[i][j];
        if (abs (sum - a[i][m]) > EPS)
            return 0;
    }

    for (int i=0; i<m; ++i)
        if (where[i] == -1)
            return INF;
    return 1;
}

int compute_rank(vector<vector<double>> A) {
    int n = A.size();
    int m = A[0].size();

    int rank = 0;
    vector<bool> row_selected(n, false);
    for (int i = 0; i < m; ++i) {
        int j;
        for (j = 0; j < n; ++j) {
            if (!row_selected[j] && abs(A[j][i]) > EPS)
                break;
        }

        if (j != n) {
            ++rank;
            row_selected[j] = true;
            for (int p = i + 1; p < m; ++p)
                A[j][p] /= A[j][i];
            for (int k = 0; k < n; ++k) {
                if (k != j && abs(A[k][i]) > EPS) {
                    for (int p = i + 1; p < m; ++p)
                        A[k][p] -= A[j][p] * A[k][i];
                }
            }
        }
    }
    return rank;
}\end{lstlisting}
\subsection{Gradient Descent}
\begin{lstlisting}
/*
Given n 3D point. Find a point from where distance to farthest of those n point is minimum.
*/
int main(){
    int n;
    cin >> n;
    double x[n],y[n],z[n];
    REP(i,n) cin >> x[i] >> y[i] >> z[i];
    double px=0.0,py=0.0,pz=0.0,alpha=1.0;
    REP(loop,100000){
        int idx=0;
        double dis=-1.0;
        REP(i,n){
            double tmp=SQ(x[i]-px)+SQ(y[i]-py)+SQ(z[i]-pz);
            if(tmp>dis) {
                dis=tmp;
                idx=i;
            }
        }
        px+=alpha*(x[idx]-px);
        py+=alpha*(y[idx]-py);
        pz+=alpha*(z[idx]-pz);
        alpha*=0.999;
    }
    cout << px << ' ' << py << ' ' << pz;
}
\end{lstlisting}
\subsection{Green Hackenbush on Trie}
\begin{lstlisting}
int trie[40 * MAX][26];
int XOR[40 * MAX][26];
int valu[40 * MAX];
int node = 1;

int add(string s) {
  int now = 1;
  stack<int>st;
  for (int i = 0; i < s.size(); i++) {
    int c = s[i] - 'a';
    if (!trie[now][c]) trie[now][c] = ++node;
    st.push(now);
    now = trie[now][c];
  }

  int nxt = now;
  int nxt_val = 0;
  for (int i = 0; i < 26; i++) nxt_val ^= XOR[now][i];
  while (!st.empty()) {
    now = st.top();
    st.pop();
    int val = 0;
    for (int i = 0; i < 26; i++) {
      if (trie[now][i] == nxt) {
        XOR[now][i] = nxt_val + 1;
      }
      val ^= XOR[now][i];
    }
    nxt_val = val;
    nxt = now;
  }
  return nxt_val;
}
\end{lstlisting}
\subsection{Grid Nim}
\begin{lstlisting}
int r,c;
scanf("%d %d",&r,&c);
int nim=0;
FOR(i,1,r){
    FOR(j,1,c){
        int tmp;
        scanf("%d",&tmp);
        if(((r-i)+(c-j))%2){
            nim^=tmp;
        }
    }
}
if(nim) printf("Case %d: win\n",tc);
else printf("Case %d: lose\n",tc);
\end{lstlisting}
\subsection{Linear Seive with Multiplicative Functions}
\begin{lstlisting}
const int maxn = 1e7;
vector <int> primes;
int spf[maxn + 5], phi[maxn + 5], NOD[maxn + 5], cnt[maxn + 5], POW[maxn + 5], SOD[maxn + 5];
bool prime[maxn + 5];

void sieve() {
    fill(prime + 2, prime + maxn + 1, 1);
    SOD[1] = NOD[1] = phi[1] = spf[1] = 1;
    for (ll i = 2; i <= maxn; i++) {
        if (prime[i]) {
            primes.push_back(i), spf[i] = i;
            phi[i] = i - 1;
            NOD[i] = 2, cnt[i] = 1;
            SOD[i] = i + 1, POW[i] = i;
        }
        for (auto p : primes) {
            if (p * i > maxn or p > spf[i]) break;
            prime[p * i] = false, spf[p * i] = p;
            if (i % p == 0) {
                phi[p * i] = p * phi[i];
                NOD[p * i] = NOD[i] / (cnt[i] + 1) * (cnt[i] + 2), cnt[p * i] = cnt[i] + 1;
                SOD[p * i] = SOD[i] / SOD[POW[i]] * (SOD[POW[i]] + p * POW[i]), POW[p * i] = p * POW[i];
                break;
            } else {
                phi[p * i] = phi[p] * phi[i];
                NOD[p * i] = NOD[p] * NOD[i], cnt[p * i] = 1;
                SOD[p * i] = SOD[p] * SOD[i], POW[p * i] = p;
            }

        }
    }
}\end{lstlisting}
\subsection{Matrix Exponentiation}
\begin{lstlisting}

struct matrix {
  vector<vector<ll>> mat;
  int n, m;
  matrix() {}
  matrix(int n, int m) : n(n), m(m), mat(n) {
    for (int i = 0; i < n; i++) mat[i] = vector<ll>(m);
  }

  void identity() { for (int i = 0; i < n; i++) mat[i][i] = 1; }
  void print() {
    for (int i = 0; i < n; i++) {
      for (int j = 0; j < n; j++) cout << mat[i][j] << " ";
      cout << "\n";
    }
  }
  vector<ll> &operator[](int i) {
    return mat[i];
  }
};

// make sure a.m == b.n
matrix operator * (matrix &a, matrix &b) {
  int n = a.n, m = b.m;
  matrix ret(n, m);
  for (int i = 0; i < n; i++) {
    for (int j = 0; j < m; j++) {
      for (int k = 0; k < a.m; k++) {
        ll val = (1ll * a[i][k] * b[k][j]) % MOD;
        ret[i][j] = (ret[i][j] + val) % MOD;
      }
    }
  }
  return ret;
}

matrix mat_exp(matrix &mat, ll p) {
  int n = mat.n, m = mat.m;
  matrix ret(n, m);
  ret.identity();
  matrix x = mat;

  while (p) {
    if (p & 1) ret = ret * x;
    x = x * x;
    p = p >> 1;
  }
  return ret;
}\end{lstlisting}
\subsection{Mobius Function}
\begin{lstlisting}
const int N = 1e6 + 5;
int mob[N];

void mobius() {
    memset(mob, -1, sizeof mob);
    mob[1] = 1;
    for (int i = 2; i < N; i++) if (mob[i]){
        for (int j = i + i; j < N; j += i) 
            mob[j] -= mob[i];
    }
}

\end{lstlisting}
\subsection{Modular Binomial Coefficients}
\begin{lstlisting}
const int N = 2e5+5;
const int mod = 1e9+7;

array <int, N+1> fact, inv, inv_fact;
void init(){
    fact[0] = inv_fact[0] = 1;
    for(int i=1; i<=N; i++){
        inv[i] = i == 1 ? 1 : (LL) inv[i - mod%i] * (mod/i + 1) % mod;
        fact[i] = (LL) fact[i-1] * i % mod;
        inv_fact[i] = (LL) inv_fact[i-1] * inv[i] % mod;
    }
}
int C(int n,int r){
    if(fact[0] != 1) init();
    return (r < 0 or r > n) ? 0 : (LL) fact[n]*inv_fact[r] % mod * inv_fact[n-r] % mod;
}
\end{lstlisting}
\subsection{Number Theoretic Transformation}
\begin{lstlisting}
#define pii     pair<LL,LL>
const LL N = 1 << 18;
const LL MOD = 786433;

vector<LL>P[N];

LL rev[N], w[N | 1], a[N], b[N], inv_n, g;

LL Pow(LL b, LL p) {
    LL ret = 1;
    while (p) {
        if (p & 1) ret = (ret * b) % MOD;
        b = (b * b) % MOD;
        p >>= 1;
    }
    return ret;
}

LL primitive_root(LL p) {
    vector<LL>factor;
    LL phi = p - 1, n = phi;

    for (LL i = 2; i * i <= n; i++) {
        if (n % i) continue;
        factor.emplace_back(i);
        while (n % i == 0) n /= i;
    }

    if (n > 1) factor.emplace_back(n);
    for (LL res = 2; res <= p; res++) {
        bool ok = true;
        for (LL i = 0; i < factor.size() && ok; i++) ok &= Pow(res, phi / factor[i]) != 1;
        if (ok) return res;
    }
    return -1;
}

void prepare(LL n) {
    LL sz = abs(31 - __builtin_clz(n));
    LL r = Pow(g, (MOD - 1) / n);
    inv_n = Pow(n, MOD - 2);
    w[0] = w[n] = 1;
    for (LL i = 1; i < n; i++) w[i] = (w[i - 1] * r) % MOD;
    for (LL i = 1; i < n; i++) rev[i] = (rev[i >> 1] >> 1) | ((i & 1) << (sz - 1));
}

void NTT(LL *a, LL n, LL dir = 0) {
    for (LL i = 1; i < n - 1; i++) if (i < rev[i]) swap(a[i], a[rev[i]]);
    for (LL m = 2; m <= n; m <<= 1) {
        for (LL i = 0; i < n; i += m) {
            for (LL j = 0; j < (m >> 1); j++) {
                LL &u = a[i + j], &v = a[i + j + (m >> 1)];
                LL t = v * w[dir ? n - n / m*j : n / m * j] % MOD;
                v = u - t < 0 ? u - t + MOD : u - t;
                u = u + t >= MOD ? u + t - MOD : u + t;
            }
        }
    }
    if (dir) for (LL i = 0; i < n; i++) a[i] = (inv_n * a[i]) % MOD;
}

vector<LL> mul(vector<LL>p, vector<LL>q) {
    LL n = p.size(), m = q.size();
    LL t = n + m - 1, sz = 1;
    while (sz < t) sz <<= 1;
    prepare(sz);

    for (LL i = 0; i < n; i++) a[i] = p[i];
    for (LL i = 0; i < m; i++) b[i] = q[i];

    for (LL i = n; i < sz; i++) a[i] = 0;
    for (LL i = m; i < sz; i++) b[i] = 0;

    NTT(a, sz);
    NTT(b, sz);
    for (LL i = 0; i < sz; i++) a[i] = (a[i] * b[i]) % MOD;
    NTT(a, sz, 1);

    vector<LL> c(a, a + sz);
    while (c.size() && c.back() == 0) c.pop_back();
    return c;
}

/*
N different number box
Number of ways to make a number by picking any number from any of the boxes
*/

vector<LL> solve(LL l, LL r) {
    if (l == r) return P[l];
    LL m = (l + r) / 2;
    return mul(solve(l, m), solve(m + 1, r));
}

int main() {
    LL m;
    cin >> m;
    for (LL i = 1; i <= m; i++) {
        LL num;
        cin >> num;
        vector<pii>v;
        LL mx = 0;
        while (num--) {
            LL typ, cnt;
            cin >> typ >> cnt;
            v.emplace_back(typ, cnt);
            mx = max(mx, typ);
        }
        P[i].resize(mx + 1);
        for (pii p : v) P[i][p.first] = p.second;
    }
    g = primitive_root(MOD);
    vector<LL>c = solve(1, m);
    for (LL i = 0; i < c.size(); i++) {
        if (c[i]) {
            cout << i << ' ' << c[i] << '\n';
        }
    }
}
\end{lstlisting}
\subsection{Pollard Rho and Factorization}
\begin{lstlisting}
// fast factorize
map <ull,int> fast_factorize(ull n){
    map <ull,int> ans;
    for(;n>1;n/=spf[n])
        ans[spf[n]]++;
    return ans;
}
inline ULL mul(ULL a,ULL b,ULL mod){
    LL ans = a * b - mod * (ULL) (1.L / mod * a * b);
    return ans + mod * (ans < 0) - mod * (ans >= (LL) mod);
}
inline ULL bigmod(ULL num,ULL pow,ULL mod){
    ULL ans = 1;
    for( ;  pow > 0;  pow >>= 1, num = mul(num, num, mod))
        if(pow&1) ans = mul(ans,num,mod);
    return ans;
}
inline bool is_prime(ULL n){
    if(n < 2 or n % 6 % 4 != 1) 
        return (n|1) == 3;
    ULL a[] = {2, 325, 9375, 28178, 450775, 9780504, 1795265022};
    ULL s = __builtin_ctzll(n-1), d = n >> s;
    for(ULL x: a){
        ULL p = bigmod(x % n, d, n), i = s;
        for( ; p != 1 and p != n-1 and x % n and i--; p = mul(p, p, n));
        if(p != n-1 and i != s)
            return false;
    }
    return true;
}
ULL get_factor(ULL n) {
    auto f = [&](LL x)  { return mul(x, x, n) + 1; };
    ULL x = 0, y = 0, t = 0, prod = 2, i = 2, q;
    for(  ; t++ %40 or gcd(prod, n) == 1;   x = f(x), y = f(f(y)) ){
        (x == y) ? x = i++, y = f(x) : 0;
        prod = (q = mul(prod, max(x,y) - min(x,y), n)) ? q : prod;
    }
    return gcd(prod, n);
}
map <ULL, int> factorize(ULL n){
    map <ULL, int> res;
    if(n < 2)   return res;
    ULL small_primes[] = {2, 3, 5, 7, 11, 13, 17, 19, 23, 29, 31, 37, 41, 43, 47, 53, 59, 61, 67, 71, 73, 79, 83, 89, 97 };
    for (ULL p: small_primes)
        for( ; n % p == 0; n /= p, res[p]++);

    auto _factor = [&](ULL n, auto &_factor) {
        if(n == 1)   return;
        if(is_prime(n)) 
            res[n]++;
        else {
            ULL x = get_factor(n);
            _factor(x, _factor);
            _factor(n / x, _factor);
        }
    };
    _factor(n, _factor);
    return res;
}
\end{lstlisting}
\subsection{Prime Counting Function}
\begin{lstlisting}
// initialize once by calling init()
#define MAXN 20000010       // initial sieve limit
#define MAX_PRIMES 2000010  // max size of the prime array for sieve
#define PHI_N 100000
#define PHI_K 100

int len = 0;  // total number of primes generated by sieve
int primes[MAX_PRIMES];
int pref[MAXN];        // pref[i] --> number of primes <= i
int dp[PHI_N][PHI_K];  // precal of yo(n,k)
bitset<MAXN> f;
void sieve(int n) {
    f[1] = true;
    for (int i = 4; i <= n; i += 2) f[i] = true;
    for (int i = 3; i * i <= n; i += 2) {
        if (!f[i]) {
            for (int j = i * i; j <= n; j += i << 1) f[j] = 1;
        }
    }
    for (int i = 1; i <= n; i++) {
        if (!f[i]) primes[len++] = i;
        pref[i] = len;
    }
}
void init() {
    sieve(MAXN - 1);
    // precalculation of phi upto size (PHI_N,PHI_K)
    for (int n = 0; n < PHI_N; n++) dp[n][0] = n;
    for (int k = 1; k < PHI_K; k++) {
        for (int n = 0; n < PHI_N; n++) {
            dp[n][k] = dp[n][k - 1] - dp[n / primes[k - 1]][k - 1];
        }
    }
}
// returns the number of integers less or equal n which are
// not divisible by any of the first k primes
// recurrence --> yo(n, k) = yo(n, k-1) - yo(n / p_k , k-1)
// for sum of primes yo(n, k) = yo(n, k-1) - p_k * yo(n / p_k , k-1)
long long yo(long long n, int k) {
    if (n < PHI_N && k < PHI_K) return dp[n][k];
    if (k == 1) return ((++n) >> 1);
    if (primes[k - 1] >= n) return 1;
    return yo(n, k - 1) - yo(n / primes[k - 1], k - 1);
}
// complexity: n^(2/3).(log n^(1/3))
long long Legendre(long long n) {
    if (n < MAXN) return pref[n];
    int lim = sqrt(n) + 1;
    int k = upper_bound(primes, primes + len, lim) - primes;
    return yo(n, k) + (k - 1);
}
// runs under 0.2s for n = 1e12
long long Lehmer(long long n) {
    if (n < MAXN) return pref[n];
    long long w, res = 0;
    int b = sqrt(n), c = Lehmer(cbrt(n)), a = Lehmer(sqrt(b));
    b = Lehmer(b);
    res = yo(n, a) + ((1LL * (b + a - 2) * (b - a + 1)) >> 1);
    for (int i = a; i < b; i++) {
        w = n / primes[i];
        int lim = Lehmer(sqrt(w));
        res -= Lehmer(w);
        if (i <= c) {
            for (int j = i; j < lim; j++) {
                res += j;
                res -= Lehmer(w / primes[j]);
            }
        }
    }
    return res;
}
\end{lstlisting}
\subsection{Seive Upto 1e9}
\begin{lstlisting}
// credit: min_25
// takes 0.5s for n = 1e9
vector<int> sieve(const int N, const int Q = 17, const int L = 1 << 15) {
  static const int rs[] = {1, 7, 11, 13, 17, 19, 23, 29};
  struct P { 
    P(int p) : p(p) {}
    int p; int pos[8];
  };
  auto approx_prime_count = [] (const int N) -> int {
    return N > 60184 ? N / (log(N) - 1.1)
                     : max(1., N / (log(N) - 1.11)) + 1;
  };

  const int v = sqrt(N), vv = sqrt(v);
  vector<bool> isp(v + 1, true);
  for (int i = 2; i <= vv; ++i) if (isp[i]) {
    for (int j = i * i; j <= v; j += i) isp[j] = false;
  }

  const int rsize = approx_prime_count(N + 30);
  vector<int> primes = {2, 3, 5}; int psize = 3;
  primes.resize(rsize);

  vector<P> sprimes; size_t pbeg = 0;
  int prod = 1; 
  for (int p = 7; p <= v; ++p) {
    if (!isp[p]) continue;
    if (p <= Q) prod *= p, ++pbeg, primes[psize++] = p;
    auto pp = P(p); 
    for (int t = 0; t < 8; ++t) {
      int j = (p <= Q) ? p : p * p;
      while (j % 30 != rs[t]) j += p << 1;
      pp.pos[t] = j / 30;
    }
    sprimes.push_back(pp);
  }

  vector<unsigned char> pre(prod, 0xFF);
  for (size_t pi = 0; pi < pbeg; ++pi) {
    auto pp = sprimes[pi]; const int p = pp.p;
    for (int t = 0; t < 8; ++t) {
      const unsigned char m = ~(1 << t);
      for (int i = pp.pos[t]; i < prod; i += p) pre[i] &= m;
    }
  }

  const int block_size = (L + prod - 1) / prod * prod;
  vector<unsigned char> block(block_size); unsigned char* pblock = block.data();
  const int M = (N + 29) / 30;

  for (int beg = 0; beg < M; beg += block_size, pblock -= block_size) {
    int end = min(M, beg + block_size);
    for (int i = beg; i < end; i += prod) {
      copy(pre.begin(), pre.end(), pblock + i);
    }
    if (beg == 0) pblock[0] &= 0xFE;
    for (size_t pi = pbeg; pi < sprimes.size(); ++pi) {
      auto& pp = sprimes[pi];
      const int p = pp.p;
      for (int t = 0; t < 8; ++t) {
        int i = pp.pos[t]; const unsigned char m = ~(1 << t);
        for (; i < end; i += p) pblock[i] &= m;
        pp.pos[t] = i;
      }
    }
    for (int i = beg; i < end; ++i) {
      for (int m = pblock[i]; m > 0; m &= m - 1) {
        primes[psize++] = i * 30 + rs[__builtin_ctz(m)];
      }
    }
  }
  assert(psize <= rsize);
  while (psize > 0 && primes[psize - 1] > N) --psize;
  primes.resize(psize);
  return primes;
}

int32_t main() {
  ios_base::sync_with_stdio(0);
  cin.tie(0);
  int n, a, b; cin >> n >> a >> b;
  auto primes = sieve(n);
  vector<int> ans;
  for (int i = b; i < primes.size() && primes[i] <= n; i += a) ans.push_back(primes[i]);
  cout << primes.size() << ' ' << ans.size() << '\n';
  for (auto x: ans) cout << x << ' '; cout << '\n';
  return 0;
}
\end{lstlisting}
\subsection{Simpson Integration}
\begin{lstlisting}
/*
    For finding the length of an arc in a range
    L = integrate(ds) from start to end of range
    where ds = sqrt(1+(d/dy(x))^2)dy
*/
const double SIMPSON_TERMINAL_EPS = 1e-12;
/// Function whose integration is to be calculated
double F(double x);
double simpson(double minx, double maxx)
{
    return (maxx - minx) / 6 * (F(minx) + 4 * F((minx + maxx) / 2.) + F(maxx));
}
double adaptive_simpson(double minx, double maxx, double c, double EPS)
{
//    if(maxx - minx < SIMPSON_TERMINAL_EPS) return 0;

    double midx = (minx + maxx) / 2;
    double a = simpson(minx, midx);
    double b = simpson(midx, maxx);

    if(fabs(a + b - c) < 15 * EPS) return a + b + (a + b - c) / 15.0;

    return adaptive_simpson(minx, midx, a, EPS / 2.) + adaptive_simpson(midx, maxx, b, EPS / 2.);
}
double adaptive_simpson(double minx, double maxx, double EPS)
{
    return adaptive_simpson(minx, maxx, simpson(minx, maxx, i), EPS);
}

\end{lstlisting}
\subsection{Stirling Numbers}
\begin{lstlisting}
//stirling number 2nd kind variation(number of ways to place n marbles in k boxes so that each box has at least x marbles)
ll solve(int marble, int box) {
  if (marble < 1ll * box * x) return 0;
  if (box == 1 && marble >= x) return 1;
  if (vis[marble][box] == cs) return dp[marble][box];
  vis[marble][box] = cs;
  ll a = ( 1ll * box * solve(marble - 1, box) ) % MOD;
  ll b = ( 1ll * box * ncr(marble - 1, x - 1) ) % MOD;
  b = (b * solve(marble - x, box - 1)) % MOD;
  ll ret = (a + b) % MOD;
  return dp[marble][box] = ret;
}
//number of ways to place n marbles in k boxes so that no box is empty
ll stir(ll n, ll k) {
  ll ret = 0;
  for (int i = 0; i <= k; i++) {
    ll v = ncr(k, i) * bigmod(i, n) % MOD;
    if ( (k - i) % 2 == 0 ) ret = (ret + v) % MOD;
    else ret = (ret - v + MOD) % MOD;
  }
  return ret;
}
\end{lstlisting}
\subsection{Subset Convolution}
\begin{lstlisting}

array<array<int, N>, b> Fh = {0}, Gh = {0}, H = {0};

for (int mask = 0; mask < N; mask++)
    Fh[__builtin_popcount(mask)][mask] = F[mask],
    Gh[__builtin_popcount(mask)][mask] = G[mask];

for (int i = 0; i < b; i++)
    for (int j = 0; j < b; j++)
        for (int mask = 0; mask < N; mask++)
            if ((mask & (1 << j)) != 0)
                Fh[i][mask] += Fh[i][mask ^ (1 << j)],
                    Gh[i][mask] += Gh[i][mask ^ (1 << j)];

for (int mask = 0; mask < N; mask++)
    for (int i = 0; i < b; i++)
        for (int j = 0; j <= i; j++)
            H[i][mask] += Fh[j][mask] * Gh[i - j][mask];

for (int i = 0; i < b; i++)
    for (int j = 0; j < b; j++)
        for (int mask = 0; mask < N; mask++)
            if ((mask & (1 << j)) != 0) H[i][mask] -= H[i][mask ^ (1 << j)];

for (int mask = 0; mask < N; mask++)
    Ans[mask] = H[__builtin_popcount(mask)][mask];
\end{lstlisting}
\subsection{Xor Basis}
\begin{lstlisting}
struct XorBasis {
    static const int sz = 64;
    array <ULL, sz> base = {0}, back;
    array <int, sz> pos;
    void insert(ULL x, int p) {
        ULL cur = 0;
        for(int i = sz - 1; ~i; i--) if (x >> i & 1) {
            if(!base[i]) {
                base[i] = x, back[i] = cur, pos[i] = p;
                break;
            } else x ^= base[i], cur |= 1ULL << i;
        }
    }
    pair <ULL, vector <int>> construct(ULL mask) {
        ULL ok = 0, x = mask;
        for(int i = sz - 1; ~i; i--)  
            if(mask >> i & 1 and base[i])
                mask ^= base[i], ok |= 1ULL << i;
        vector <int> ans;
        for(int i = 0; i < sz; i++) if(ok >> i & 1) {
            ans.push_back(pos[i]);
            ok ^= back[i];
        }
        return {x ^ mask, ans};
    } 
};\end{lstlisting}
\section{String Algorithms}
\subsection{Aho Corasick}
\begin{lstlisting}
struct state {
  int len, par, link, next_lif;
  ll val;
  int next[26];
  char p_char;
  bool lif;
  state(int par = -1, char p_char = '$', int len = 0) : par(par), p_char(p_char), len(len) {
    lif = false;
    link = 0;
    next_lif = 0;
    val = 0;
    memset(next, 0, sizeof next);
  }
};
vector<state>aho;
inline void add_str(const string &s, ll val) {
  int now = 0;
  for (int i = 0; i < s.size(); i++) {
    int c = s[i] - 'a';
    if (!aho[now].next[c]) {
      aho[now].next[c] = (int)aho.size();
      aho.emplace_back(now, s[i], aho[now].len + 1);
    }
    now = aho[now].next[c];
  }
  aho[now].lif = true;
  aho[now].val = val;
}
inline void push_link() {
  queue<int>q;
  q.push(0);
  while (!q.empty()) {
    int cur = q.front();
    int link = aho[cur].link;
    q.pop();

    aho[cur].next_lif = aho[link].lif ? link : aho[link].next_lif;
    for (int c = 0; c < 26; c++) {
      if (aho[cur].next[c]) {
        aho[ aho[cur].next[c] ].link = cur ? aho[link].next[c] : 0;
        q.push( aho[cur].next[c] );
      } else aho[cur].next[c] = aho[link].next[c];
    }
    aho[cur].val += aho[link].val;
  }
}
inline int count(string &s) {
  int now = 0, ret = 0;
  for (int i = 0; i < s.size(); i++) {
    now = aho[now].next[s[i] - 'a'];
    ret += aho[now].val;
  }
  return ret;
}
struct dynamic_aho {
  aho_corasick ac[20];
  vector<string> dict[20];
  dynamic_aho() {
    for (int i = 0; i < 20; i++) {
      ac[i].aho.clear();
      dict[i].clear();
    }
  }
  void add_str(string &s) {
    int idx = 0;
    for (; idx < 20 && !ac[idx].aho.empty(); idx++) {}
    ac[idx] = aho_corasick();
    ac[idx].add_str(s, 1), dict[idx].pb(s);
    for (int i = 0; i < idx; i++) {
      for (string x : dict[i]) {
        ac[idx].add_str(x, 1);
        dict[idx].pb(x);
      }
      ac[i].aho.clear(), dict[i].clear();
    }
    ac[idx].push_link();
  }
  inline int count(string &s) {
    int ret = 0;
    for (int i = 0; i < 20; i++) {
      if (!ac[i].aho.empty()) ret += ac[i].count(s);
    }
    return ret;
  }
};
int arr[MAX];
int main() {
  fastio;
  dynamic_aho add, del;
  int m;
  cin >> m;
  while (m--) {
    int type;
    string s;
    cin >> type >> s;
    if (type == 1) add.add_str(s);
    else if (type == 2) del.add_str(s);
    else cout << add.count(s) - del.count(s) << "\n" << flush;
  }
}
\end{lstlisting}
\subsection{Double Hash}
\begin{lstlisting}
ostream& operator << (ostream& os, pll hash) {
  return os << "(" << hash.ff << ", " << hash.ss << ")";
}

pll operator + (pll a, ll x)     {return pll(a.ff + x, a.ss + x);}
pll operator - (pll a, ll x)     {return pll(a.ff - x, a.ss - x);}
pll operator * (pll a, ll x)     {return pll(a.ff * x, a.ss * x);}
pll operator + (pll a, pll x)    {return pll(a.ff + x.ff, a.ss + x.ss);}
pll operator - (pll a, pll x)    {return pll(a.ff - x.ff, a.ss - x.ss);}
pll operator * (pll a, pll x)    {return pll(a.ff * x.ff, a.ss * x.ss);}
pll operator % (pll a, pll m)    {return pll(a.ff % m.ff, a.ss % m.ss);}

pll base(1949313259, 1997293877);
pll mod(2091573227, 2117566807);

pll power (pll a, ll p) {
  if (!p) return pll(1, 1);
  pll ans = power(a, p / 2);
  ans = (ans * ans) % mod;
  if (p % 2) ans = (ans * a) % mod;
  return ans;
}

pll inverse(pll a) {
  return power(a, (mod.ff - 1) * (mod.ss - 1) - 1);
}

pll inv_base = inverse(base);

pll val;
vector<pll> P;

void hash_init(int n) {
  P.resize(n + 1);
  P[0] = pll(1, 1);
  for (int i = 1; i <= n; i++) P[i] = (P[i - 1] * base) % mod;
}

///appends c to string
pll append(pll cur, char c) {
  return (cur * base + c) % mod;
}

///prepends c to string with size k
pll prepend(pll cur, int k, char c) {
  return (P[k] * c + cur) % mod;
}

///replaces the i-th (0-indexed) character from right from a to b;
pll replace(pll cur, int i, char a, char b) {
  cur = (cur + P[i] * (b - a)) % mod;
  return (cur + mod) % mod;
}

///Erases c from the back of the string
pll pop_back(pll hash, char c) {
  return (((hash - c) * inv_base) % mod + mod) % mod;
}

///Erases c from front of the string with size len
pll pop_front(pll hash, int len, char c) {
  return ((hash - P[len - 1] * c) % mod + mod) % mod;
}

///concatenates two strings where length of the right is k
pll concat(pll left, pll right, int k) {
  return (left * P[k] + right) % mod;
}

///Calculates hash of string with size len repeated cnt times
///This is O(log n). For O(1), pre-calculate inverses
pll repeat(pll hash, int len, ll cnt) {
  pll mul = (P[len * cnt] - 1) * inverse(P[len] - 1);
  mul = (mul % mod + mod) % mod;
  pll ret = (hash * mul) % mod;

  if (P[len].ff == 1) ret.ff = hash.ff * cnt;
  if (P[len].ss == 1) ret.ss = hash.ss * cnt;
  return ret;
}

ll get(pll hash) {
  return ( (hash.ff << 32) ^ hash.ss );
}

struct hashlist {
  int len;
  vector<pll> H, R;

  hashlist() {}
  hashlist(string &s) {
    len = (int)s.size();
    hash_init(len);
    H.resize(len + 1, pll(0, 0)), R.resize(len + 2, pll(0, 0));
    for (int i = 1; i <= len; i++) H[i] = append(H[i - 1], s[i - 1]);
    for (int i = len; i >= 1; i--) R[i] = append(R[i + 1], s[i - 1]);
  }
  
  /// 1-indexed
  inline pll range_hash(int l, int r) {
    int len = r - l + 1;
    return ((H[r] - H[l - 1] * P[len]) % mod + mod) % mod;
  }

  inline pll reverse_hash(int l, int r) {
    int len = r - l + 1;
    return ((R[l] - R[r + 1] * P[len]) % mod + mod) % mod;
  }

  inline pll concat_range_hash(int l1, int r1, int l2, int r2) {
    int len_2 = r2 - l2 + 1;
    return concat(range_hash(l1, r1), range_hash(l2, r2), len_2);
  }

  inline pll concat_reverse_hash(int l1, int r1, int l2, int r2) {
    int len_1 = r1 - l1 + 1;
    return concat(reverse_hash(l2, r2), reverse_hash(l1, r1), len_1);
  }
};
\end{lstlisting}
\subsection{Faster Hash Table}
\begin{lstlisting}
#include <ext/pb_ds/assoc_container.hpp>
using namespace __gnu_pbds;
struct custom_hash {
  const ll rnd = chrono::high_resolution_clock::now().time_since_epoch().count();
  static unsigned long long hash_f(unsigned long long x) {
    x += 0x9e3779b97f4a7c15;
    x = (x ^ (x >> 30)) * 0xbf58476d1ce4e5b9;
    x = (x ^ (x >> 27)) * 0x94d049bb133111eb;
    return x ^ (x >> 31);
  }
  ll operator() (ll x) const { return hash_f(x) ^ rnd; }
  // static int combine_hash(pii hash) { return hash.ff * 31 + hash.ss; }
  // static ll combine_hash(pll hash) { return (hash.ff << 32) ^ hash.ss; }
  // ll operator() (pll x) const { 
  //   x.ff = hash_f(x.ff) ^ rnd; x.ss = hash_f(x.ss) ^ rnd;
  //   return combine_hash(x); 
  // }
};
gp_hash_table<ll, int, custom_hash> mp;
unordered_map<ll, int, custom_hash> mp;
\end{lstlisting}
\subsection{Knuth Morris Pratt}
\begin{lstlisting}
vector<int> prefix_function(string s) {
  int n = (int)s.length();
  vector<int> pi(n);
  for (int i = 1; i < n; i++) {
    int j = pi[i - 1];
    while (j > 0 && s[i] != s[j]) j = pi[j - 1];
    if (s[i] == s[j]) j++;
    pi[i] = j;
  }
  return pi;
}
\end{lstlisting}
\subsection{Manacher}
\begin{lstlisting}
vector<int> d1(n); ///odd palindromes
for (int i = 0, l = 0, r = -1; i < n; i++) {
  int k = (i > r) ? 1 : min(d1[l + r - i], r - i + 1);
  while (0 <= i - k && i + k < n && s[i - k] == s[i + k]) {
    k++;
  }
  d1[i] = k--;
  if (i + k > r) {
    l = i - k;
    r = i + k;
  }
}

vector<int> d2(n); ///even palindromes
for (int i = 0, l = 0, r = -1; i < n; i++) {
  int k = (i > r) ? 0 : min(d2[l + r - i + 1], r - i + 1);
  while (0 <= i - k - 1 && i + k < n && s[i - k - 1] == s[i + k]) {
    k++;
  }
  d2[i] = k--;
  if (i + k > r) {
    l = i - k - 1;
    r = i + k ;
  }
}\end{lstlisting}
\subsection{Palindromic Tree}
\begin{lstlisting}
struct state {
  int len, link;
  map<char, int> next;
};
state st[MAX];
int id, last;
string s;
ll ans[MAX];
void init() {
  for (int i = 0; i <= id; i++) {
    st[i].len = 0; st[i].link = 0;
    st[i].next.clear(); ans[i] = 0;
  }
  st[1].len = -1; st[1].link = 1;
  st[2].len = 0; st[2].link = 1;
  id = 2; last = 2;
}
void extend(int pos) {
  while (s[pos - st[last].len - 1] != s[pos]) last = st[last].link;
  int newlink = st[last].link;
  char c = s[pos];
  while (s[pos - st[newlink].len - 1] != s[pos]) newlink = st[newlink].link;
  if (!st[last].next.count(c)) {
    st[last].next[c] = ++id;
    st[id].len = st[last].len + 2;
    st[id].link = (st[id].len == 1 ? 2 : st[newlink].next[c]);
    ans[id] += ans[st[id].link];
    if (st[id].len > 2) {
      int l = st[id].len / 2 + (st[id].len % 2 ? 1 : 0);
      if (h.range_hash(pos - st[id].len + 1, pos - st[id].len + l) == h.reverse_hash(pos - st[id].len + 1, pos - st[id].len + l)) ans[id]++;
    }
  }
  last = st[last].next[c];
}
\end{lstlisting}
\subsection{String Match FFT}
\begin{lstlisting}
//find occurrences of t in s where '?'s are automatically matched with any character
//res[i + m - 1] = sum_j=0 to m - 1_{s[i + j] * t[j] * (s[i + j] - t[j])
vector<int> string_matching(string &s, string &t) {
  int n = s.size(), m = t.size();
  vector<int> s1(n), s2(n), s3(n);
  for(int i = 0; i < n; i++) s1[i] = s[i] == '?' ? 0 : s[i] - 'a' + 1; //assign any non zero number for non '?'s
  for(int i = 0; i < n; i++) s2[i] = s1[i] * s1[i];
  for(int i = 0; i < n; i++) s3[i] = s1[i] * s2[i];
  vector<int> t1(m), t2(m), t3(m);
  for(int i = 0; i < m; i++) t1[i] = t[i] == '?' ? 0 : t[i] - 'a' + 1;
  for(int i = 0; i < m; i++) t2[i] = t1[i] * t1[i];
  for(int i = 0; i < m; i++) t3[i] = t1[i] * t2[i];
  reverse(t1.begin(), t1.end());
  reverse(t2.begin(), t2.end());
  reverse(t3.begin(), t3.end());
  vector<int> s1t3 = multiply(s1, t3);
  vector<int> s2t2 = multiply(s2, t2);
  vector<int> s3t1 = multiply(s3, t1);
  vector<int> res(n);
  for(int i = 0; i < n; i++) res[i] = s1t3[i] - s2t2[i] * 2 + s3t1[i];
  vector<int> oc;
  for(int i = m - 1; i < n; i++) if(res[i] == 0) oc.push_back(i - m + 1);
  return oc;
}
\end{lstlisting}
\subsection{Suffix Array}
\begin{lstlisting}
vector<vector<int> >c;
vector<int>sort_cyclic_shifts(string const& s) {
  int n = s.size();
  const int alphabet = 256;
  vector<int> p(n), cnt(alphabet, 0);
  c.clear(); c.emplace_back(); c[0].resize(n);
  for (int i = 0; i < n; i++) cnt[s[i]]++;
  for (int i = 1; i < alphabet; i++) cnt[i] += cnt[i - 1];
  for (int i = 0; i < n; i++) p[--cnt[s[i]]] = i;
  c[0][p[0]] = 0;
  int classes = 1;
  for (int i = 1; i < n; i++) {
    if (s[p[i]] != s[p[i - 1]]) classes++;
    c[0][p[i]] = classes - 1;
  }
  vector<int> pn(n), cn(n); cnt.resize(n);
  for (int h = 0; (1 << h) < n; h++) {
    for (int i = 0; i < n; i++) {
      pn[i] = p[i] - (1 << h);
      if (pn[i] < 0) pn[i] += n;
    }
    fill(cnt.begin(), cnt.end(), 0);
    /// radix sort
    for (int i = 0; i < n; i++) cnt[c[h][pn[i]]]++;
    for (int i = 1; i < classes; i++) cnt[i] += cnt[i - 1];
    for (int i = n - 1; i >= 0; i--) p[--cnt[c[h][pn[i]]]] = pn[i];
    cn[p[0]] = 0; classes = 1;
    for (int i = 1; i < n; i++) {
      pii cur = {c[h][p[i]], c[h][(p[i] + (1 << h)) % n]};
      pii prev = {c[h][p[i - 1]], c[h][(p[i - 1] + (1 << h)) % n]};
      if (cur != prev) ++classes;
      cn[p[i]] = classes - 1;
    }
    c.push_back(cn);
  }
  return p;
}
vector<int> suffix_array_construction(string s) {
  s += "!";
  vector<int> sorted_shifts = sort_cyclic_shifts(s);
  sorted_shifts.erase(sorted_shifts.begin());
  return sorted_shifts;
}
/// compare two suffixes starting at i and j with length 2^k
int compare(int i, int j, int n, int k) {
  pii a = {c[k][i], c[k][(i + 1 - (1 << k)) % n]};
  pii b = {c[k][j], c[k][(j + 1 - (1 << k)) % n]};
  return a == b ? 0 : a < b ? -1 : 1;
}
int lcp(int i, int j) {
  int log_n = c.size() - 1;
  int ans = 0;
  for (int k = log_n; k >= 0; k--) {
    if (c[k][i] == c[k][j]) {
      ans += 1 << k;
      i += 1 << k; j += 1 << k;
    }
  }
  return ans;
}
vector<int> lcp_construction(string const& s, vector<int> const& p) {
  int n = s.size();
  vector<int> rank(n, 0);
  for (int i = 0; i < n; i++) rank[p[i]] = i; 
  int k = 0;
  vector<int> lcp(n, 0);
  for (int i = 0; i < n; i++) {
    if (rank[i] == n - 1) {
      k = 0;
      continue;
    }
    int j = p[rank[i] + 1];
    while (i + k < n && j + k < n && s[i + k] == s[j + k]) k++;
    lcp[rank[i]] = k;
    if (k) k--;
  }
  return lcp;
}
//kth lexicographically smallest substring (considering duplicates)
int arr[MAX], lg[MAX], mn[MAX][25];
vector<int>p, lcp;
string s;
int k;

// sparse table for min in lcp goes here

///find the rightmost position where get(l,r) > val
int khoj(int l, int r, int val) {
  int lo = l + 1, hi = r, ret = -1;
  while (lo <= hi) {
    int mid = lo + (hi - lo) / 2;
    if (get(l, mid - 1) > val) {
      ret = mid; lo = mid + 1;
    } else hi = mid - 1;
  }
  return ret;
}
int done[MAX];
int arr[MAX];
int main() {
  fastio;
  cin >> s >> k;
  p = suffix_array_construction(s);
  lcp = lcp_construction(s, p);
  build();
  int n = s.size();
  int milaisi = 0;
  for (int i = 0; i < n; i++) {
    milaisi += done[i];
    int len = n - p[i];
    int cur = milaisi;
    /// cur = i ? lcp[i-1] : 0; this can replace all the milaisi and done parts
    while (cur < len) {
      int r = khoj(i, n - 1, cur);
      int koyta, milabo;
      if (r == -1) {
        milabo = len - cur;
        koyta = 1;
      } else {
        milabo = get(i, r - 1) - cur;
        koyta = r - i + 1;
      }
      if (koyta * milabo < k) k -= (koyta * milabo);
      else {
        int d = k / koyta; int m = k % koyta;
        if (!m) {
          cout << s.substr(p[i], cur + d) << "\n";
          return 0;
        } else {
          cout << s.substr(p[i], cur + d + 1) << "\n";
          return 0;
        }
      }
      if (r == -1) break;
      done[r + 1] -= milabo;
      cur = get(i, r - 1);
      milaisi += milabo;
    }
  }
  cout << "No such line.\n";
}
\end{lstlisting}
\subsection{Suffix Automaton}
\begin{lstlisting}
struct state {
  int len, link;
  map<char, int> next;
  bool is_clone;
  int first_pos;
  vector<int>inv_link;
};
state st[2 * MAX];
int mn[2 * MAX], mx[2 * MAX];
int sz, last;
void sa_init() {
  st[0].len = 0;
  st[0].link = -1;
  st[0].next.clear();
  sz = 1;
  last = 0;
}
void sa_extend(char c) {
  int cur = sz++;
  st[cur].len = st[last].len + 1;
  st[cur].first_pos = st[cur].len - 1;
  st[cur].is_clone = false;
  st[cur].next.clear();
  ///for lcs of n strings
  // mn[cur] = st[cur].len;
  int p = last;
  while (p != -1 && !st[p].next.count(c)) {
    st[p].next[c] = cur;
    p = st[p].link;
  }
  if (p == -1) {
    st[cur].link = 0;
  } else {
    int q = st[p].next[c];
    if (st[p].len + 1 == st[q].len) {
      st[cur].link = q;
    } else {
      int clone = sz++;
      st[clone].len = st[p].len + 1;
      st[clone].next = st[q].next;
      st[clone].link = st[q].link;
      st[clone].first_pos = st[q].first_pos;
      st[clone].is_clone = true;
      ///for lcs of n strings
      // mn[clone] = st[clone].len;
      while (p != -1 && st[p].next[c] == q) {
        st[p].next[c] = clone;
        p = st[p].link;
      }
      st[q].link = st[cur].link = clone;
    }
  }
  last = cur;
}
void radix_sort() {
  for (int i = 0; i < sz; i++) cnt[st[i].len]++;
  for (int i = 1; i < sz; i++) cnt[i] += cnt[i - 1];
  for (int i = 0; i < sz; i++) order[--cnt[st[i].len]] = i;
}
// after constructing the automaton
for (int v = 1; v < sz; v++) {
  st[st[v].link].inv_link.push_back(v);
}
// output all positions of occurrences
void output_all_occurrences(int v, int P_length) {
  if (!st[v].is_clone)
    cout << st[v].first_pos - P_length + 1 << endl;
  for (int u : st[v].inv_link)
    output_all_occurrences(u, P_length);
}
//lcs of two strings
string lcs (string S, string T) {
  sa_init();
  for (int i = 0; i < S.size(); i++) sa_extend(S[i]);
  int v = 0, l = 0, best = 0, bestpos = 0;
  for (int i = 0; i < T.size(); i++) {
    while (v && !st[v].next.count(T[i])) {
      v = st[v].link; l = st[v].len;
    }
    if (st[v].next.count(T[i])) {
      v = st[v].next[T[i]]; l++;
    }
    if (l > best) best = l; bestpos = i;
  }
  return T.substr(bestpos - best + 1, best);
}
//lcs of n strings
void add_str(string s) {
  for (int i = 0; i < sz; i++) mx[i] = 0;
  int v = 0, l = 0;
  for (int i = 0; i < s.size(); i++) {
    while (v && !st[v].next.count(s[i])) {
      v = st[v].link; l = st[v].len;
    }
    if (st[v].next.count(s[i])) {
      v = st [v].next[s[i]]; l++;
    }
    mx[v] = max(mx[v], l);
  }
  for (int i = sz - 1; i > 0; i--) mx[st[i].link] = max(mx[st[i].link], mx[i]);
  for (int i = 0; i < sz; i++) mn[i] = min(mn[i], mx[i]);
}
int lcs() {
  int ret = 0;
  for (int i = 0; i < sz; i++) ret = max(ret, mn[i]);
  return ret;
}
string s[15];
int arr[MAX];
int main() {
  fastio;
  int n = 0;
  while (cin >> s[n]) n++;
  sa_init();
  for (int i = 0; i < s[0].size(); i++) sa_extend(s[0][i]);
  for (int i = 1; i < n; i++) add_str(s[i]);
  cout << lcs() << "\n";
}
\end{lstlisting}
\subsection{Z Algorithm}
\begin{lstlisting}
vector<int> calcz(string s) {
    int n = s.size();
    vector<int> z(n);
    int l, r; l = r = 0;
    for (int i = 1; i < n; i++) {
        if (i > r) {
            l = r = i;
            while (r < n && s[r] == s[r - l]) r++;
            z[i] = r - l; r--;
        } else {
            int k = i - l;
            if (z[k] < r - i + 1) z[i] = z[k];
            else {
                l = i;
                while (r < n && s[r] == s[r - l]) r++;
                z[i] = r - l; r--;
            }
        }
    }
    return z;
}
\end{lstlisting}
\section{Tricks}
\subsection{Array Compression}
\begin{lstlisting}
void compress(int n) {
  vector<int> v;
  for (int i = 0; i < n; i++) v.pb(arr[i]);
  sort(v.begin(), v.end());
  v.erase(unique(v.begin(), v.end()), v.end());
  for (int i = 0; i < n; i++) {
    int zipped = lower_bound(v.begin(), v.end(), arr[i]) - v.begin() + 1;
    arr[i] = zipped;
  }
} 
\end{lstlisting}
\subsection{Different Cumulative Sum}
\begin{lstlisting}
///cost = sum of (i * a[i]) where i starts from the beginning for every range
///sum[] = prefix sum of value[i]
///isum[] = prefix sum of i*value[i]
ll cost(int i, int j) {
  ll ret = isum[j];
  if (i) ret -= isum[i - 1];
  ll baad = sum[j];
  if (i) baad -= sum[i - 1];
  return ret - i * baad;
} 
\end{lstlisting}
\subsection{Fractional Binary Search}
\begin{lstlisting}
/**
Given a function f and n, finds the smallest fraction p / q in [0, 1] or [0,n]
such that f(p / q) is true, and p, q <= n.
Time: O(log(n))
**/
struct frac { long long p, q; };
bool f(frac x) {
 return 6 + 8 * x.p >= 17 * x.q + 12;
}
frac fracBS(long long n) {
  bool dir = 1, A = 1, B = 1;
  frac lo{0, 1}, hi{1, 0}; // Set hi to 1/0 to search within [0, n] and {1, 1} to search within [0, 1]
  if (f(lo)) return lo;
  assert(f(hi)); //checking if any solution exists or not
  while (A || B) {
    long long adv = 0, step = 1; // move hi if dir, else lo
    for (int si = 0; step; (step *= 2) >>= si) {
      adv += step;
      frac mid{lo.p * adv + hi.p, lo.q * adv + hi.q};
      if (abs(mid.p) > n || mid.q > n || dir == !f(mid)) {
        adv -= step; si = 2;
      } 
    }
    hi.p += lo.p * adv;
    hi.q += lo.q * adv;
    dir = !dir;
    swap(lo, hi);
    A = B; B = !!adv;
  }
  return dir ? hi : lo;
}
\end{lstlisting}
\subsection{Time Count}
\begin{lstlisting}
clock_t start,finish;
double timespent;
start=clock();
//Here Comes your Code/Function
finish=clock();
timespent=(double)(finish-start)/CLOCKS_PER_SEC;
cout << timespent << '\n';
\end{lstlisting}
\end{multicols*}
\begin{multicols*}{2}
\newpage
\section{Equations and Formulas}
\subsection{Catalan Numbers}
$\displaystyle C_n=\frac{1}{n+1}{2n \choose n}$
$\displaystyle C_0=1,C_1=1\text{ and }C_n=\sum \limits_{k=0}^{n-1}C_k C_{n-1-k}$ \\
The number of ways to completely parenthesize $n$+$\displaystyle 1$ factors. \\
The number of triangulations of a convex polygon with $n$+$\displaystyle 2$ sides (i.e. the number of partitions of polygon into disjoint triangles by using the diagonals). \\
The number of ways to connect the $\displaystyle 2n$ points on a circle to form $n$ disjoint i.e. non-intersecting chords. \\
The number of rooted full binary trees with $n$+$\displaystyle 1$ leaves (vertices are not numbered). A rooted binary tree is full if every vertex has either two children or no children. \\
Number of permutations of $\displaystyle {1, …, n}$ that avoid the pattern $\displaystyle 123$ (or any of the other patterns of length $3$); that is, the number of permutations with no three-term increasing sub-sequence. For $n = 3$, these permutations are $\displaystyle 132,\ 213,\ 231,\ 312$ and $321.$

\subsection{Stirling Numbers First Kind}
The Stirling numbers of the first kind count permutations according to their number of cycles (counting fixed points as cycles of length one). \\
$S(n,k)$ counts the number of permutations of $n$ elements with $\displaystyle \displaystyle k$ disjoint cycles. \\
$S(n,k)=(n-1) \cdot S(n-1,k)+S(n-1,k-1),$ \(where,\; S(0,0)=1,S(n,0)=S(0,n)=0\)
$\displaystyle \displaystyle\sum_{k=0}^{n}S(n,k) = n!$ \\
The unsigned Stirling numbers may also be defined algebraically, as the coefficient of the rising factorial:
\[\displaystyle x^{\bar{n}} = x(x+1)...(x+n-1) = \sum_{k=0}^{n}{ S(n, k) x^k}\]
Lets $[n, k]$ be the stirling number of the first kind, then

\[\displaystyle \bigl[\!\begin{smallmatrix} n \\ n\ -\ k \end{smallmatrix}\!\bigr] = \sum_{0 \leq i_1 < i_2 < i_k < n}{i_1i_2....i_k.}\]

\subsection{Stirling Numbers Second Kind}
Stirling number of the second kind is the number of ways to partition a set of n objects into k non-empty subsets. \\
$S(n,k)=k \cdot S(n-1,k)+S(n-1,k-1)$, \(where \; S(0,0)=1,S(n,0)=S(0,n)=0\)
$S(n,2)=2^{n-1}-1$ 
$S(n,k) \cdot k!$ = number of ways to color $n$ nodes using colors from $\displaystyle 1$ to $\displaystyle \displaystyle k$ such that each color is used at least once. \\
An $r$-associated Stirling number of the second kind is the number of ways to partition a set of $n$ objects into $\displaystyle \displaystyle k$ subsets, with each subset containing at least $r$ elements. It is denoted by $S_r( n , k )$ and obeys the recurrence relation. $\displaystyle \displaystyle S_r(n+1,k) = k S_r(n,k) + \binom{n}{r-1} S_r(n-r+1,k-1)$ \\ 
Denote the n objects to partition by the integers $\displaystyle 1, 2, …., n$. Define the reduced Stirling numbers of the second kind, denoted $S^d(n, k)$, to be the number of ways to partition the integers $\displaystyle 1, 2, …., n$ into k nonempty subsets such that all elements in each subset have pairwise distance at least d. That is, for any integers i and j in a given subset, it is required that $|i - j| \geq d$. It has been shown that these numbers satisfy, \(S^d(n, k) = S(n - d + 1, k - d + 1), n \geq k \geq d\)
\subsection{Other Combinatorial Identities}
$\displaystyle \displaystyle {n \choose k}=\frac{n}{k}{n-1 \choose k-1}$ \\
$\displaystyle \sum \limits_{i= 0}^k{n+i \choose i}= \sum \limits_{i= 0}^k{n+i \choose n} = {n+k+1 \choose k}$ \\
$\displaystyle \ n,r \in N, n > r, \sum \limits_{i=r}^n{i \choose r}={n+1 \choose r+1}$ \\
If $\displaystyle P(n)=\sum_{k=0}^{n}{n \choose k} \cdot Q(k)$, then,
\[Q(n)=\sum_{k=0}^{n}(-1)^{n-k}{n \choose k} \cdot P(k)\] \\
If $\displaystyle P(n)=\sum_{k=0}^{n}(-1)^{k}{n \choose k} \cdot Q(k)$ , then,
\[Q(n)=\sum_{k=0}^{n}(-1)^{k}{n \choose k} \cdot P(k)\]

\subsection{Different Math Formulas}
\begin{multline*}
\displaystyle (1 - x)(1 - x ^ 2)(1 - x^3)\cdots = \\ 1 - x - x^2 + x^5 + x^7 - x^{12} - x^{15} + x^{22} + x^{26} - \cdots.
\end{multline*}

The exponents $\displaystyle 1, 2, 5, 7, 12,\cdots$ on the right hand side are given by the formula $g_k = \frac{k(3k - 1)}{2}$ for $\displaystyle \displaystyle k = 1, -1, 2, -2, 3, \cdots$ and are called (generalized) pentagonal numbers.
It is useful to find the partition number in $O(n \sqrt{n})$ \\ \\
Let $\displaystyle a$ and $b$ be coprime positive integers, and find integers $\displaystyle a{\prime}$ and $b{\prime}$ such that $\displaystyle aa{\prime} \equiv 1 \mod b$ and $bb{\prime} \equiv 1 \mod a$. Then the number of representations of a positive integers (n) as a non negative linear combination of $\displaystyle a$ and $b$ is 
\begin{multline*}
\displaystyle \frac{n}{ab}-\Big\{\frac{b{\prime} n}{a}\Big\}-\Big\{\frac{a{\prime} n}{b}\Big\} + 1
\end{multline*}
\subsection{GCD and LCM}
if $m$ is any integer, then $\displaystyle \gcd(a + m {\cdot} b, b) = \gcd(a, b)$ \\
The gcd is a multiplicative function in the following sense: if $\displaystyle a_1$ and $\displaystyle a_2$ are relatively prime, then $\displaystyle \gcd(a_1 \cdot a_2, b) = \gcd(a_1, b) \cdot \gcd(a_2,b )$. \\
$\displaystyle \gcd(a, \operatorname{lcm}(b, c)) = \operatorname{lcm}(\gcd(a, b), \gcd(a, c))$. \\
$\displaystyle \operatorname{lcm}(a, \gcd(b, c)) = \gcd(\operatorname{lcm}(a, b), \operatorname{lcm}(a, c))$. \\
For non-negative integers $\displaystyle a$ and $b$, where $\displaystyle a$ and $b$ are not both zero, $\displaystyle \gcd({n^a} - 1, {n^b} - 1) = n^{\gcd(a,b)} - 1$ \\
$\displaystyle \gcd(a, b) = \displaystyle \sum_{k|a \, \text{and} \, k|b} {\phi(k)}$ \\
$\displaystyle \displaystyle \sum_{i=1}^n [\gcd(i, n) = k] = { \phi{\left(\frac{n}{k}\right)}}$ \\
$\displaystyle \displaystyle \sum_{k=1}^n \gcd(k, n) = \displaystyle \sum_{d|n} d \cdot {\phi{\left(\frac{n}{d}\right)}}$ \\
$\displaystyle \displaystyle \sum_{k=1}^n x^{\gcd(k,n)} = \displaystyle \sum_{d|n} x^d \cdot {\phi{\left(\frac{n}{d}\right)}}$ \\
$\displaystyle \displaystyle \sum_{k=1}^n \frac{1}{\gcd(k, n)} = \displaystyle \sum_{d|n} \frac{1}{d} \cdot {\phi{\left(\frac{n}{d}\right)}} = \frac{1}{n} \displaystyle \sum_{d|n} d \cdot \phi(d)$ \\
$\displaystyle \displaystyle \sum_{k=1}^n \frac{k}{\gcd(k, n)} = \frac{n}{2} \cdot \displaystyle \sum_{d|n} \frac{1}{d} \cdot {\phi{\left(\frac{n}{d}\right)}} = \frac{n}{2} \cdot \frac{1}{n} \cdot \displaystyle \sum_{d|n} d \cdot \phi(d)$ \\
$\displaystyle \displaystyle \sum_{k=1}^n \frac{n}{\gcd(k, n)} = 2 * \displaystyle \sum_{k=1}^n \frac{k}{\gcd(k, n)} - 1$, for $n > 1$ \\
$\displaystyle \displaystyle \sum_{i=1}^n \sum_{j=1}^n [\gcd(i, j) = 1] = \displaystyle \sum_{d=1}^n \mu(d) \lfloor {\frac{n}{d} \rfloor}^2$ \\
$\displaystyle \displaystyle \sum_{i=1}^n \displaystyle\sum_{j=1}^n \gcd(i, j) = \displaystyle \sum_{d=1}^n \phi(d) \lfloor {\frac{n}{d} \rfloor}^2$ \\
$\displaystyle \sum_{i=1}^n \sum_{j=1}^n i \cdot j[\gcd(i, j) = 1] = \sum_{i=1}^n \phi(i)i^2$ \\
$\displaystyle F(n) = \displaystyle \sum_{i=1}^n \displaystyle \sum_{j=1}^n \operatorname{lcm}(i, j) = \displaystyle \sum_{l=1}^n {\left(\frac{\left( 1 + \lfloor{\frac{n}{l} \rfloor} \right) \left( \lfloor{\frac{n}{l} \rfloor} \right)} {2} \right)}^2 \displaystyle \sum_{d|l} \mu(d)ld$ \\
\end{multicols*}
\end{document}
